\documentclass[a4paper]{article}

\title{Synthèse des articles lus}

\author{Stéphane Kastenbaum}

\date{2018-06-25}

\begin{document}

\maketitle

\section{Problématique}

\subsection{Escape}

Escape est un projet de logiciel pour modéliser les évacuations de villes en cas
de crise (comme l'explosion d'une usine, ou bien l'inondation d'un pont).  Ces
crises peuvent modifier l'espace de ville, comme interdire l'accès de certaines
zone aux piétons, ou bien couper un pont. Ce qui évidemment gêne l'évacuation de
la population, et doit donc être pris en compte dans la modélisation.

Escape sera donc une simulation multi-agents où les agents seront des piétons ou
des automobilistes qui se déplaceront dans l'espace, tandis qu'il ne connaitront
pas parfaitement cet espace.

\subsection{Représentation spatiale}

On doit donc décider d'une façon de modéliser l'espace permettant aux agents de
se déplacer dans celui-ci. L'espace étant, par exemple, le réseau routier,
l'ensemble des rues que les piétons peuvent emprunter ou bien le réseau de
transports en commun.

Il faut que notre modélisation permettent aux agents de savoir où ils sont, et
quel chemin il doivent prendre pour aller là où ils veulent.

  \subsubsection{Superposition de plusieurs réseaux}

  Souvent dans une ville il y a plusieurs réseaux de transports qui coexistent,
  par exemple le réseau routier et le réseau de transports en commun. Les
  relations entre les réseaux sont variés, un piétons peut par exemple prendre
  le réseau des transports en commun ou décider de marcher jusqu'à sa
  destination, tandis qu'un automobiliste est obligé d'emprunter le réseau
  routier ou bien de garer sa voiture est de devenir un piéton.

  De plus, les différents réseaux peuvent être affectés différemment par les
  aléas. Une rue peut, par exemple, ne pas être praticable pour les
  automobiliste, mais être traversable par les piétons.

\subsection{Relations spatio-temporels}

L'espace peut être amené à se modifier au cours de la simulation, la crue peut
monter et bloquer d'autres ponts, ou un nuage toxique peut se déplacer par
exemple. Il est donc important de prévoir une modélisation qui peut facilement
être modifiée. Les modifications que l'on voudra apporter peuvent prendre
plusieurs formes : un chemin reliant deux espaces peut être coupé ; un espace
peut être à éviter et donc tous les chemins menant à cet endroit doivent être
enlevés ou un chemin peut devenir plus difficile à traverser (le terrain peut
devenir boueux par exemple) et il faudra noter que la vitesse des agents sur ce
terrain est ralentie.

\subsection{Diversités des connaissances}

Dans une ville, il se peut que les agents n'aient pas les mêmes connaissances de
l'espace, par exemple certaines personnes peuvent connaître certaines rues ou
des raccourcis que d'autres ne connaissent pas. Cela va évidemment influer sur
les choix des agents, ainsi la modélisation se doit de pouvoir représenter cette
diversité.

  \subsubsection{Modification dynamique de la connaissance par la perception}

  Au cours de la simulation, les agents vont percevoir l'espace et leurs
  connaissances de l'espace vont être modifié en conséquence. Il faut donc
  prévoir que les connaissances de chacun peuvent être modifiées.

  \subsubsection{Partage de la connaissance entre agents}

  Parfois deux agents n'aillant pas la même connaissance de l'espace vont se
  rencontrer et ils vont mettre en commun leurs connaissance. Il faut réfléchir
  à une façon de fusionner les connaissance et de juger quel individu va
  convaincre l'autre que sa connaissance est la plus exacte et doit être
  conservée.

\subsection{Contraintes techniques}

La ville étant un espace grand et complexe, et le nombre d'agents gigantesque,
il va falloir faire très attention à la taille mémoire que va prendre la
modélisation et à la quantité de calculs nécessaire à la simulation notamment,
les calculs de plus court chemins que vont faire les agents.

\section{État de l'art}

Il existe plusieurs façon de représenter l'espace : en matrice comme dans
\cite{batty1999} qui sont utiles pour représenter les étendues, les
\emph{maptree} proposés par \cite{worboys2012} qui se focalisent sur la
représentation des frontières, Region Connection Calculus (RCC-8)
\cite{randell1992} définit les régions et leurs connections topologiques (se
touchent, se superposent, à l'intérieur de...) ou bien les graphes spatiaux qui
représentent l'espace comme un agrégat de nodes reliés par des arêtes, qui sont
très pratiques pour considérer les chemins. Les graphes sont utilisés dans comme
représentions de l'espace dans beaucoup de modélisations, comme le montre
\cite{dale2010}.

Notre choix se porte donc sur une représentation de l'espace avec un graphe car
nous nous intéressons principalement aux chemins et aux déplacements de nos
agents.

Il y a des multiples façon de représenter l'espace avec des graphes. Pour le
réseau de rues par exemple on peut considérer que les intersections des rues
sont des nodes et relier chaque nodes si il y a un tronçon de rue entre les deux
[ref]. On peut aussi prendre le graphe dual et avoir chaque tronçon de rue comme
nodes et deux nodes sont reliés si elles se croisent [ref]. Ou bien encore
chaque rue est une node et relier deux nodes si les rues s'intersectent [ref].
On peut faire de même pour le réseau routier ou celui des transport en commun.

Pour représenter l'espace, on peut aussi parler de granularité (#TOCHANGE), les
\emph{stratified map space} dans \cite{stell1998} sont des représentations de
l'espace selon plusieurs degrés de précisions sémantiques et géographiques.
C'est particulièrement utile pour représenter simplement un espace qui a
beaucoup de détails. En effet il est parfois plus intéressant d'enlever de la
complexité à une modélisation pour la rendre plus lisible. C'est le principe de
généralisation en cartographique, par exemple \cite{mackaness1993} utilise la
théorie des graphes pour déterminer quels attributs peuvent être enlevés pour
une représentation plus générale.

Les \emph{stratified map space} ressemblent un peu au \emph{multi level overlay
graphs}\cite{holzer2009}. Ce sont des graphes auquel on ajoute une couche
supplémentaire qui est une sélection des nodes les plus importantes pour le
calcul de chemins, on relie ces nodes avec le plus court chemin entre elles cela
permet ensuite de retrouver le plus court chemin entre toutes les nodes plus
rapidement. Pour notre utilisations, \cite{bruera2008} qui rends ces
\emph{overlay graphs} dynamiques, est particulièrement intéressant.

Malgré les différentes façon de décrire le temps une semble faire consensus,
celle développée dans \cite{allen1985}. Il définit une algèbre avec les
intervalles de temps comme objet de base et avec ceux-ci définit des relations
et tout ce qu'il faut pour avoir une algèbre temporel complète.

Lorsqu'on parle de dynamisme spatio-temporel on voit souvent des articles qui
s'intéresse aux processus modifiant l'espaces. Certains, comme
\cite{claramunt1995}, décrivent les processus et évènements comme des entités.
Ce sont des \emph{Temporal Geographic Information Systems}(T-GIS),
\cite{siabato2018} fait une étude de tous travaux fait sur les T-GIS. D'autres
comme \cite{delmondo2011} et \cite{costes2015} se penchent sur la représentation
des relations entre les objets dans des espaces et des instants différents.

Il semble que nous devrons faire quelque chose dans le genre de
\cite{jguirim2015}, où auteur décrit comment chaque individus a un profil
différent. Dans leurs modélisations, tous les individus sont des piétons, mais
ils marchent à des vitesses différentes, et certains ont accès au réseau de
transports en commun.

En ce qui concerne la modélisations des connaissances, on a peu d'articles
traitant ce sujet de la façon dont on voudrait le traiter. Il existe des études
sur les niveaux de connaissances, comme dans \cite{stern1988} qui fait une étude
pour mesurer le niveau de connaissance de la ville qu'ont les conducteurs
professionnels par rapport aux conducteurs amateurs. Il y a aussi
\cite{quinn2018} qui s'intéresse au rapport qu'on les gens aux risques naturels
près de chez eux, en fonction de leur attachement à leur quartier. Cela ne
correspond pas à ce qu'on veut faire, on veut plutôt faire quelque chose dans le
genre de \cite{kuipers1978} qui crée un modèle reproduisant la façon dont les
individus se représentent l'espace. L'idée étant que les gens retiennent surtout
les chemins entre leurs points d'intérêts et la position relative de certains
lieux par rapport à des points de références (un monument visible de loin par
exemple).

Il semblerait qu'il y a peu d'articles détaillant la façon dont on peut
modéliser la diversité des connaissances de l'espace qu'ont les individus.
Idéalement on aimerait une façon de modéliser la différence entre le graphe
représentant l'espace et celui représentant la connaissance qu'a un individu de
cette espace. Comme les individus peuvent avoir une partie de connaissance en
commun il faudrait pouvoir le modéliser.

Comme la simulation multi-agents va souvent demander aux agents de décider où
aller, il faut trouver une technique pour ne pas avoir à refaire tout le calcul
de plus court chemin à chaque fois qu'un individu cherche comment aller quelque
part. Par exemple, les \emph{multi levels overlay graphs} de \cite{holzer2009}
peuvent être une piste pour résoudre ce problème.

\section{Bibliographie}

\bibliographystyle{plain-fr} \bibliography{biblio.bib}

\end{document}
