\documentclass[a4paper]{article}

\title{Synthèse des articles lus}
\author{Stéphane Kastenbaum}

\begin{document}

\maketitle

\section{Problématique}
\subsection{Escape}

Escape est un projet de logiciel pour modéliser les évacuations de villes en cas
d'une crise (comme l'explosion d'une usine, ou bien l'inondation d'un pont).
Ces crises peuvent modifier l'espace de ville, comme interdire l'accès de
certaines zone aux piétons, ou bien couper un pont. Ce qui évidemment gêne
l'évacuation de la population, et doit donc être pris en compte dans la
modélisation.

Escape sera donc une simulation multi-agents où les agents seront des piétons
ou des automobilistes qui se déplaceront dans l'espace, tandis qu'il ne
connaitront pas parfaitement cet espace.

\subsection{Représentation spatiales}

On doit donc décider d'une façon de modéliser l'espace permettant aux agents de
se déplacer dans celui-ci. L'espace étant, par exemple, le réseau routier. Il
faut que les agents puissent savoir où ils sont, et quel chemin il doivent
prendre pour aller là où ils veulent.

\subsection{Relations spatio-temporels}

L'espace peut être amené à se modifier au cours de la simulation, la crue peut
monter et bloquer d'autres ponts, ou un nuage toxique peut se déplacer par
exemple. Il est donc important de prévoir une modélisation qui peut facilement
être modifiée. Les modifications que l'on voudra apporter peuvent prendre
plusieurs forme, un chemin reliant deux espaces peut être coupé, un espace
peut être à éviter et donc tous les chemins menant à cet endroit doivent être
enlevés. Un chemin peut aussi devenir plus difficile à traverser (le terrain
peut devenir boueux par exemple), il faudra donc noter que la vitesse des agents
sur ce terrain est ralentie.

\subsection{Diversités des connaissances}

Dans une ville, il se peut que les agents n'aient pas les mêmes connaissances
de l'espace, par exemple certaines personnes peuvent connaître certaines rues
ou des raccourcis que d'autres ne connaissent pas. Cela va évidemment influer
sur les choix des agents, ainsi la modélisation se doit de pouvoir représenter
cette diversité.

\subsubsection{Modification dynamique de la connaissance par la perception}

Au cours de la simulation, les agents vont percevoir l'espace et leurs
connaissances de l'espace vont être modifié en conséquence. Il faut donc prévoir
que les connaissances de chacun peuvent être modifiée.

\subsection{Partage de la connaissance entre agents}

Parfois deux agents n'aillant pas la même connaissance de l'espace vont se
rencontrer et ils vont mettre en commun leurs connaissance. Il faut réfléchir
à une façon de fusionner les connaissance et de juger quelle connaissance doit
être conservée.

\subsection{Limitations techniques}

La ville étant un espace grand est complexe, et le nombre d'agents gigantesque,
il va falloir faire très attention à la taille mémoire que va prendre la
modélisation et à la quantité de calculs nécessaire à la simulation.

\section{État de l'art}

\end{document}
