\documentclass[a4paper]{article}

\usepackage[utf8]{inputenc}
\usepackage[T1]{fontenc}
\usepackage{lmodern}
\usepackage[french]{babel}
\usepackage{xcolor}
\usepackage{tikz}
\usepackage{nameref}
\usepackage{tabularx}
\usepackage{comments}

% \newcommand{\comment}[1]{}

\title{Synthèse des articles lus}

\author{Stéphane Kastenbaum}

\date{2018-07-05}

\begin{document}

\maketitle

\section{Problématique}

\subsection{Escape}

Escape est un projet visant à définir des outils informatiques pour modéliser
les évacuations de villes en cas de crise (comme l'explosion d'une usine, ou
bien la rupture d'une digue). Ces situations de crises peuvent modifier l'espace
de la ville, comme empêcher l'accès de certaines zone aux piétons, ou bien
couper un pont.  Ceci gêne l'évacuation de la population, et doit être pris en
compte dans la modélisation.

L'objectif d'Escape est de pouvoir valider des plans d'évacuation comme un Plan
Communal de Sauvegarde (PCS). Un PCS est un document décrivant l'organisation à
adopter face aux risques de désastre naturel, industriel ou sanitaire. Il est
écrit par des experts avec l'équipe de la mairie. Ils y indique tout ce qu'il
faut faire pour informer la population sur les risques, l'alerter en cas de
crise, la guider lors de l'évacuation, et l'assister si besoin est. Pour
valider un PCS, Escape doit vérifier la faisabilité et l'efficacité des
dispositions proposés dans le plan.

Escape s'intéresse principalement à tester les dispositions prises lors de la
crise. Escape cherche entre autres à estimer le temps qu'il faudrait pour que la
population évacue la ville dans différent scénarios et selon les dispositions
prises dans le PCS. L'équipe de la mairie peut ainsi tester différentes actions,
les évaluer et les comparer, pour faire un meilleur PCS.

Pour simuler l'évacuation de la ville, la méthode mise en œuvre est un système
multi-agents. Un système multi-agents est un programme informatique où l'on
simule plusieurs agents autonomes. Un agent est une entité situé dans un
environnement, auquel on associe un comportement le rendant capable de prendre
des décisions et d'interagir avec son environnement. Dans Escape, les agent sont
les civils (piétons ou automobilistes) et les agents de gestion de crise.

Dans les systèmes multi-agents, les agents évoluent sur une structure
informatique modélisant l'espace. Il existe plusieurs type de structures pour
modéliser l'espace. En effet, différentes structures de modélisations permettent
de simuler les différentes caractéristiques de l'espace.

Pour que les agents prennent des décisions, nous devons leur fournir des
connaissances et un algorithme qui, avec ces connaissances, indique la décision
à prendre. Nous nous intéressons ici à la structure modélisant la connaissance
de l'espace. On appelle connaissance de l'espace toutes les informations
relative à son environment que l'individu a à sa disposition. Il y une grande
diversité de connaissance, et des individus avec des connaissances différentes
font des choix différents. Ainsi la modélisation se doit de pouvoir représenter
cette diversité.

Dans la section "\nameref{etat}" nous verrons les différentes structures
utilisés dans la littérature pour modéliser l'espace et pour modéliser la
connaissance de l'espace.

On fera bien attention à différencier l'espace réels, la modélisation
informatique de cet espace, la connaissance qu'ont les individus de l'espace, et
la modélisation de cette connaissance. Aussi, on parlera «~d'individus~» lorsque
l'on fera référence aux entités réel et «~d'agents~» pour désigner leur
modélisation informatique (résumé dans la table \ref{tab:vocab}).

\begin{table}
  \noindent\makebox[\textwidth]{
  \begin{tabular}{|c||c|c|c|}
    % \hline
    % {}&Espace objectif&Espace subjectif&Entités \\
    \hline
    Environnement réel & Espace & Connaissance de l'espace & Individu \\
    \hline
    Modélisation informatique & Modélisation de l'espace & Modélisation de la
    connaissance &
    Agents\\
    \hline
  \end{tabular}
  }
  \caption{Le vocabulaire utilisé}
  \label{tab:vocab}
\end{table}

Notre objectif est de modéliser la réalité le plus fidèlement possible. Pour
cela il faut bien comprendre ce que l'on veut modéliser. Dans la section qui
suit nous allons décrire les caractéristiques de l'espace et de la connaissance
de cette espace que nous voulons modéliser.

\subsection{L'espace et la connaissance de l'espace}

On définit l'espace comme étant l'étendue de terre où les individus peuvent se
déplacer. Dans le contexte de l'évacuation de ville, il est plus pertinent de
définir l'espace de la ville comme l'ensemble des modes de transports
disponibles dans la ville, car c'est principalement via ces réseaux que les
individus vont évacuer la ville. C'est-à-dire le réseau pédestre, le réseau
routier, les transports en commun, etc.

La connaissance de l'espace est l'ensemble des informations qu'a un individu à
propos des modes de transport disponible dans sa ville et leurs modalités.  On y
inclut l'ensemble des rues qu'il connaît, les distances, les routes, les lignes
des transports en commun, leurs horaires etc.

Les habitants d'une ville n'ont pas la même connaissance de l'espace. Par
exemple, certaines personnes connaissent des rues ou des raccourcis que d'autres
ne connaissent pas, comme illustré sur la figure~\ref{fig:espace}. Autrement
dit, les individus ont une connaissance personnelle de l'espace. Cette
connaissance peut être partielle et/ou inexacte. On peut supposer qu'il existe
autant de connaissance de l'espace possible que d'individus. Cependant cette
diversité des connaissances est contrebalancée par le fait que les individus ont
accès à des sources d'information communes comme les cartes, les panneaux
indicateurs dans les rues, et les connaissances des autres individus lorsqu'ils
discutent.

Alors il y a pleins de sources d'info différentes des cartes : gmap, cartes
papiers, cartes sur GPS. Bouche à oreilles,\comment{PLUS !!!}

\tikzset{legende/.style={draw, rectangle, rounded corners, fill=blue!20}}
\begin{figure}
  \centering
  \begin{tikzpicture}[scale=2]
    \newcommand{\drawMissing}[3]{
      \draw[dotted, thin] (#1) -- (#2) node[midway, #3] {?};
    }

    \newcommand{\drawSpace}[1]{
      \node[draw, circle] (C) at (0,2) {Maison 1};
      \node[draw, circle] (D) at (0.5,0) {Maison 2};
      \node[draw, circle] (A) at (1,1) {Travail};
      \node[draw, circle] (B) at (0,1) {};
      \node[legende] at (0.5, -0.75) {#1};
    }

    \begin{scope}[shift={(0,0)}]
      \drawSpace{Espace objectif}
      \draw (A) to (C);
      \draw (B) to (C);
      \draw (B) to (D);
      \draw (A) to (D);
    \end{scope}

    \begin{scope}[shift={(-1.5,-4)}]
      \drawSpace{Connaissance de l'individu 1}
      \draw (A) to (C);
      \draw (B) to (C);
      \draw (B) to (D);
      \drawMissing{A}{D}{right}
    \end{scope}

    \begin{scope}[shift={(1.5,-4)}]
      \drawSpace{Connaissance de l'individu 2}
      \drawMissing{A}{C}{above right}
      \draw (B) to (C);
      \draw (B) to (D);
      \draw (A) to (D);
    \end{scope}

  \end{tikzpicture}
  \caption{La différence entre l'espace objectif, et la connaissance de
  l'espace. L'individu 1 ne connait pas le chemin entre la maison 2 et le lieu
  de travail, et réciproquement pour l'individu 2}
  \label{fig:espace}
\end{figure}

Souvent dans une ville plusieurs réseaux de transports coexistent, par exemple
le réseau routier et le réseau de transports en commun. Ces réseau ne sont pas
forcément accessibles par tous les individus, et certains individus peuvent
emprunter plusieurs réseaux consécutivement. Par exemple, un piéton peut
emprunter les transports en commun puis décider de marcher jusqu'à sa
destination, tandis qu'un automobiliste est obligé d'emprunter le réseau routier
ou bien de garer sa voiture et de devenir un piéton.

Ces types de déplacements multimodaux apportent une caractérisation des
individus supplémentaire. Non seulement les individus ne connaissent pas la
ville de la même façon, mais leur accès aux moyens de transports de la ville est
différent.

De plus, les différents réseaux peuvent être affectés différemment par les
aléas. Une rue peut, par exemple, ne pas être praticable pour les automobiliste
à cause de la pluie, mais être traversable par les piétons.

Finalement, il y a deux concepts à modéliser : l'espace, autrement dit
l'ensemble des modes de transports de la ville ; et la connaissance de cet
espace, c'est-à-dire toutes les informations qu'ont les individus sur cet
espace. Cependant, lors d'une crise naturelle ou industrielle, l'espace est
modifié par les aléas et les individus vont prendre progressivement connaissance
de la modification de cet espace, donc leurs connaissances sont modifiées. Les
modifications de l'espace et des connaissances sont dynamiques d'où la nécessité
de s'intéresser à l'évolution temporelle de l'espace et des connaissances.

\subsection{Évolution temporelle}

L'espace de la ville peut être amené à se modifier au cours de la crise. Une
modification de cet espace est tout changement affectant le déplacement des
individus. Les modifications peuvent prendre plusieurs formes. Il y a des
modifications qui empêche complètement l'accès à certaines zones, e.g., la
propagation d'un gaz toxique, dû à l'explosion d'une usine. D'autre qui coupe
une arête d'un réseau de transport, e.g., l'effondrement d'un pont. Il y a aussi
des modifications plus légères qui n'empêchent pas totalement l'accès à une
zone, mais le rends plus difficile, e.g., le terrain devient boueux.

La modification peut avoir plusieurs causes, le cas le plus courant dans notre
cas étant que l'aléa altère une partie du réseau. La modification de l'espace
peut aussi être dûs aux actions de l'homme. Typiquement, un PCS peut
recommander de fermer un pont par soucis de sécurité.

Lorsque l'espace est modifié pendant la crise, les individus ne sont à priori
pas immédiatement au courant de cette modification. Pour l'intégrer à leur
connaissance, il faut soit qu'ils la voient, soit que l'information leur
parvienne par une source quelconque.

La multitudes des sources d'information disponibles et leur non-harmonisation
provoque des conflits. En fait les sources d'information ne se mettent pas
toutes à jour automatiquement. Et les individus ne regardent pas tout le temps
les sources d'information.\comment{expliquer pourquoi les sources d'info ne sont
pas harmonisés} C'est-à-dire que certains individus reçoivent des informations
contradictoires de plusieurs sources différentes. Dans cette situation, les
individus doivent réfléchir pour choisir quelle information ils doivent
conserver.

Parfois les individus se regroupent. Dans un groupe plusieurs personnes peuvent
ne pas avoir la même connaissance de l'espace. Comme leurs décisions sont prises
à partir des informations qu'ils possèdent, leur différence de connaissances
entraine une différence de décision. Cependant, comme ils sont dans le même
groupe une seule décision peut être prise. L'un des individus doit convaincre
les autres que sa décision est la meilleure car sa connaissance est plus proche
de la réalité.\comment{Retravailler ce para}

Ces évolutions temporelles amènent les individus à devoir sans cesse réévaluer
leurs choix, puisqu'un chemin autrefois jugé optimal doit être revu si il est
affecté par l'aléa. De plus les individus peuvent apprendre de nouvelles
informations, ce qui les obligent à réévaluer leurs choix à la lumières de
celles-ci. Pour modéliser les prises de décisions des individus, on doit faire
un certains nombre de calculs. La fréquence des modifications de l'espace et des
connaissances de l'espace pose la question de la complexité à prendre en compte
pour résoudre le problème.

\subsection{Contraintes techniques}

La ville est un espace grand et complexe, et le nombre d'agents est gigantesque.
Pour se donner un ordre d'idée, à Rouen on compte 110 169 habitants sur une
superficie de 21 km$^2$. De fait il faut faire très attention à la taille
mémoire que va prendre la modélisation.

De plus, les agents appliquent souvent des algorithmes particulièrement long sur
des structures de données complexe tel que la recherche de plus court chemins.
Il faut veiller à ce que la modélisation choisie permette de faire tourner ces
algorithmes en un temps raisonnable.

\subsection{Conclusion}

Escape est une application faite pour simuler des évacuation de ville à l'aide
d'un système multi agents. On veut modéliser l'espace et l'ensemble des
informations que peut avoir un individu à propos de cet espace. Dans le but de
faire une modélisation fidèle nous avons décrit les caractéristiques de l'espace
que nous voulons conserver. Nous avons évoqué la complexité des réseaux, des
déplacements multimodaux, les différences d'informations qu'ont les individus,
et de l'évolution temporelle de l'espace et des connaissances. Un point à ne pas
négliger pour produire une modélisation fidèle est la complexité du problème. De
fait nos capacités de calculs et d'espace mémoire ne sont pas infinies. Dans la
prochaine section nous allons faire un état de l'art des structures
informatiques pour modéliser l'espace et la connaissance de l'espace.

\section{État de l'art}
\label{etat}

Il existe plusieurs façon de représenter l'espace : en matrice comme dans
\cite{batty1999} qui sont utiles pour représenter les étendues, les
\emph{maptree} proposés par \cite{worboys2012} qui se focalisent sur la
représentation des frontières, Region Connection Calculus (RCC-8)
\cite{randell1992} définit les régions et leurs connections topologiques (se
touchent, se superposent, à l'intérieur de...) ou bien les graphes spatiaux qui
représentent l'espace comme un agrégat de nœuds reliés par des arêtes, qui sont
très pratiques pour considérer les chemins. Les graphes sont utilisés comme
représentions de l'espace dans beaucoup de modélisations, comme le montre
\cite{dale2010}.

Notre choix se porte donc sur une représentation de l'espace avec un graphe car
nous nous intéressons principalement aux chemins et aux déplacements de nos
agents.

Il y a des multiples façon de représenter l'espace avec des graphes. Pour le
réseau de rues par exemple on peut considérer que les intersections des rues
sont des nœuds et relier chaque nœud si il y a un tronçon de rue entre les
deux.\comment{Ajouter un référence} On peut aussi prendre le graphe dual et
avoir chaque tronçon de rue comme nœuds et deux nœuds sont reliés si les deux
tronçons se croisent. Ou bien encore chaque rue est un nœud et relier deux nœuds
si les rues s'intersectent. On peut faire de même pour le réseau routier ou
celui des transport en commun.

\begin{figure}
  \centering
  \begin{tikzpicture}[scale=1]
    \begin{scope}
      \node[draw, circle] (A) at (0,4) {A};
      \node[draw, circle] (B) at (0,2) {B};
      \node[draw, circle] (C) at (0,0) {C};
      \node[draw, circle] (D) at (2,4) {D};
      \node[draw, circle] (E) at (2,2) {E};
      \node[draw, circle] (F) at (2,0) {F};
      \node[legende] at (1,-1) {Graphe primal};

      \draw (A) -- (B) node[left, midway] {$\alpha$};
      \draw (C) -- (B) node[left, midway] {$\beta$};
      \draw (E) -- (B) node[above, midway] {$\gamma$};
      \draw (D) -- (E) node[right, midway] {$\delta$};
      \draw (F) -- (E) node[right, midway] {$\epsilon$};
    \end{scope}

    \begin{scope}[shift={(-3,-5)}]
      \node[draw, circle] (al) at (-0.5,3) {$\alpha$};
      \node[draw, circle] (be) at (-0.5,1) {$\beta$};
      \node[draw, circle] (ga) at (1,2) {$\gamma$};
      \node[draw, circle] (de) at (2.5,3) {$\delta$};
      \node[draw, circle] (ep) at (2.5,1) {$\epsilon$};
      \node[legende] at (1,0) {Graphe dual};

      \draw (al) -- (be) node[left, midway] {B};
      \draw (al) -- (ga) node[above right, midway] {B};
      \draw (ga) -- (be) node[below right, midway] {B};
      \draw (ga) -- (de) node[above left, midway] {E};
      \draw (ga) -- (ep) node[below left, midway] {E};
      \draw (de) -- (ep) node[right, midway] {E};
    \end{scope}

    \begin{scope}[shift={(3, -3)}]
      \node[draw, circle] (ab) at (-0.5,0) {$\alpha\beta$};
      \node[draw, circle] (ga) at (1,0) {$\gamma$};
      \node[draw, circle] (de) at (2.5,0) {$\delta\epsilon$};
      \node[legende] at (1,-2) {Graphe des rues};

      \draw (ab) -- (ga) node[above, midway] {B};
      \draw (de) -- (ga) node[above, midway] {E};
    \end{scope}
  \end{tikzpicture}
  \caption{Comparaisons des graphes primal, dual et des rues, du même réseau des
  rues. Les lettres grecs représentent des tronçons de rues, et les lettres
  latines représentent les intersections.}
\end{figure}

Pour représenter l'espace, on peut aussi parler de granularité,\comment{Pas
esthétique il faudrait changer} les \emph{stratified map space} dans
\cite{stell1998} sont des représentations de l'espace selon plusieurs degrés de
précisions sémantiques et géographiques. C'est particulièrement utile pour
représenter simplement un espace qui a beaucoup de détails. En effet il est
parfois plus intéressant d'enlever de la complexité à une modélisation pour la
rendre plus lisible. C'est le principe de généralisation en cartographique, par
exemple \cite{mackaness1993} utilise la théorie des graphes pour déterminer
quels attributs peuvent être enlevés pour une représentation plus générale.

Les \emph{stratified map space} ressemblent un peu au \emph{multi level overlay
graphs}\cite{holzer2009}. Ce sont des graphes auquel on ajoute une couche
supplémentaire qui est une sélection des nœuds les plus importants pour le
calcul de chemins, on relie ces nœuds avec le plus court chemin entre eux cela
permet ensuite de retrouver le plus court chemin entre tous les nœuds plus
rapidement. Pour notre utilisation, \cite{bruera2008} qui rend ces \emph{overlay
graphs} dynamiques, est particulièrement intéressant.

Malgré les différentes façon de décrire le temps une semble faire consensus,
celle développée dans \cite{allen1985}. Il définit une algèbre avec les
intervalles de temps comme objet de base et avec ceux-ci définit des relations
et tout ce qu'il faut pour avoir une algèbre temporelle complète.

Lorsqu'on parle de dynamique spatio-temporelle on voit souvent des articles qui
s'intéresse aux processus modifiant l'espace. Certains, comme
\cite{claramunt1995} décrivent les processus et évènements comme des entités.
Ce sont des \emph{Temporal Geographic Information Systems}(T-GIS),
\cite{siabato2018} fait une étude de tous travaux fait sur les T-GIS. D'autres
comme \cite{delmondo2011} et \cite{costes2015} se penchent sur la représentation
des relations entre les objets dans des espaces et des instants différents.

Il semble que nous devrons faire quelque chose dans le genre de
\cite{jguirim2015} où l'auteur décrit comment chaque individu a un profil
différent. Dans leurs modélisations, tous les individus sont des piétons, mais
ils marchent à des vitesses différentes, et certains ont accès au réseau de
transports en commun.

En ce qui concerne la modélisations des connaissances, on ait peu d'articles
traitant ce sujet de la façon dont on voudrait le traiter. Il existe des études
sur les niveaux de connaissances, comme dans \cite{stern1988} qui fait une étude
pour mesurer le niveau de connaissance de la ville qu'ont les conducteurs
professionnels par rapport aux conducteurs amateurs. Il y a aussi
\cite{quinn2018} qui s'intéresse au rapport qu'on les gens aux risques naturels
près de chez eux, en fonction de leur attachement à leur quartier. Cela ne
correspond pas à ce qu'on veut faire, on veut plutôt faire quelque chose dans le
genre de \cite{kuipers1978} qui crée un modèle reproduisant la façon dont les
individus se représentent l'espace. L'idée étant que les gens retiennent surtout
les chemins entre leurs points d'intérêts et la position relative de certains
lieux par rapport à des points de références (un monument visible de loin par
exemple).

Il semblerait qu'il y a peu d'articles détaillant la façon dont on peut
modéliser la diversité des connaissances de l'espace qu'ont les individus.
Idéalement on aimerait une façon de modéliser la différence entre le graphe
représentant l'espace et celui représentant la connaissance qu'a un individu de
cet espace. Comme les individus peuvent avoir une partie de connaissance en
commun il faudrait pouvoir la modéliser.

Comme la simulation multi-agents va souvent demander aux agents de décider où
aller, il faut trouver une technique pour ne pas avoir à refaire tout le calcul
de plus court chemin à chaque fois qu'un individu cherche comment aller quelque
part. Par exemple, les \emph{multi levels overlay graphs} de \cite{holzer2009}
peuvent être une piste pour résoudre ce problème.

\section{Bibliographie}

\bibliographystyle{plain} \bibliography{biblio.bib}

\end{document}
