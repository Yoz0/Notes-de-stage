\documentclass[a4paper]{article}

\title{Synthèse des articles lus}
\author{Stéphane Kastenbaum}

\begin{document}

\maketitle

\section{Problématique}

\subsection{Escape}

Escape est un projet de logiciel pour modéliser les évacuations de villes en cas
de crise (comme l'explosion d'une usine, ou bien l'inondation d'un pont).  Ces
crises peuvent modifier l'espace de ville, comme interdire l'accès de certaines
zone aux piétons, ou bien couper un pont. Ce qui évidemment gêne l'évacuation
de la population, et doit donc être pris en compte dans la modélisation.

Escape sera donc une simulation multi-agents où les agents seront des piétons
ou des automobilistes qui se déplaceront dans l'espace, tandis qu'il ne
connaitront pas parfaitement cet espace.

\subsection{Représentation spatiale}

On doit donc décider d'une façon de modéliser l'espace permettant aux agents de
se déplacer dans celui-ci. L'espace étant, par exemple, le réseau routier,
l'ensemble des rues que les piétons peuvent emprunter ou bien le réseau de
transports en commun.

Il faut que notre modélisation permettent aux agents de savoir où ils sont, et
quel chemin il doivent prendre pour aller là où ils veulent.

\subsection{Relations spatio-temporels}

L'espace peut être amené à se modifier au cours de la simulation, la crue peut
monter et bloquer d'autres ponts, ou un nuage toxique peut se déplacer par
exemple. Il est donc important de prévoir une modélisation qui peut facilement
être modifiée. Les modifications que l'on voudra apporter peuvent prendre
plusieurs formes : un chemin reliant deux espaces peut être coupé ; un espace
peut être à éviter et donc tous les chemins menant à cet endroit doivent être
enlevés ou un chemin peut devenir plus difficile à traverser (le terrain peut
devenir boueux par exemple) et il faudra noter que la vitesse des agents sur ce
terrain est ralentie.

\subsection{Diversités des connaissances}

Dans une ville, il se peut que les agents n'aient pas les mêmes connaissances
de l'espace, par exemple certaines personnes peuvent connaître certaines rues
ou des raccourcis que d'autres ne connaissent pas. Cela va évidemment influer
sur les choix des agents, ainsi la modélisation se doit de pouvoir représenter
cette diversité.

  \subsubsection{Modification dynamique de la connaissance par la perception}

  Au cours de la simulation, les agents vont percevoir l'espace et leurs
  connaissances de l'espace vont être modifié en conséquence. Il faut donc
  prévoir que les connaissances de chacun peuvent être modifiée.

  \subsubsection{Partage de la connaissance entre agents}

  Parfois deux agents n'aillant pas la même connaissance de l'espace vont se
  rencontrer et ils vont mettre en commun leurs connaissance. Il faut réfléchir
  à une façon de fusionner les connaissance et de juger quelle connaissance
  doit être conservée.

\subsection{Contraintes techniques}

La ville étant un espace grand est complexe, et le nombre d'agents gigantesque,
il va falloir faire très attention à la taille mémoire que va prendre la
modélisation et à la quantité de calculs nécessaire à la simulation.

\section{État de l'art}

Dans le domaine il y a plusieurs façon de représenter l'espace en bit map [ref]
qui sont utiles pour représenter les étendues, les map-tree [ref] qui se
focalisent sur la représentation des frontières, Region Connection Calculus
(RCC-8) [ref] définit les régions et leurs connections topologiques (se
touchent, se superposent, à l'intérieur de...) ou bien les graphes spatiaux
[ref] qui représente l'espace comme un agrégat de nodes reliés par des arêtes,
qui est très pratiques pour considérer les chemins.

Notre choix se porte donc sur une représentation de l'espace avec des nodes car
nous nous intéressons principalement aux chemins et aux déplacements de nos
agents.

Il y a plusieurs façon de représenter l'espace avec des graphes. Pour le réseau
de rues par exemple on peut considérer que les intersections de rues sont des
nodes et relier chaque nodes si il y a un tronçon de rue entre les deux [ref].
On peut aussi prendre le graphe dual et avoir chaque tronçon de rue comme nodes
et deux nodes sont reliés si elles se croisent [ref]. Ou bien encore chaque rue
est une node et reliés deux nodes si les rues s'intersectent [ref]. On peut
faire de même pour le réseau routier ou celui des transport en commun.

On peut aussi parler de granularité, les \emph{stratified map space} dans
[stell1998] sont des représentations de l'espace selon plusieurs degrés de
précisions sémantiques et géographiques. C'est particulièrement utile pour
représenter simplement un espace qui a beaucoup de détails. En effet il est
parfois plus intéressant d'enlever de la complexité à une modélisation pour la
rendre plus lisible.

Les \emph{stratified map space} ressemblent un peu au \emph{multi level overlay
graphs}[ref] qui sont des graphes auquel on ajoute une couche supplémentaire
qui est une sélection des nodes les plus importantes pour le calcul de chemins,
on calcul ensuite le plus court chemin entre ces nodes, cela permet ensuite de
retrouver le plus court chemin entre toutes les nodes plus rapidement.

\end{document}
