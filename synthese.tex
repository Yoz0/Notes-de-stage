\documentclass[a4paper]{article}

\usepackage[utf8]{inputenc}
\usepackage[T1]{fontenc}
\usepackage{lmodern}
\usepackage[french]{babel}
\usepackage{xcolor}
\usepackage{tikz}
\usepackage{nameref}
\usepackage{tabularx}
 \usepackage{comments}

% \newcommand{\comment}[1]{}

\title{Synthèse des articles lus}

\author{Stéphane Kastenbaum}

\date{2018-07-05}

\begin{document}

\maketitle

\section{Problématique}

\subsection{Escape}

Escape est un projet visant à définir des outils informatiques pour modéliser
les évacuations de villes en cas de crise (comme l'explosion d'une usine, ou
bien la rupture d'une digue). Ces situations de crise peuvent modifier l'espace
de la ville, comme empêcher l'accès de certaines zones aux piétons, ou bien
couper un pont. Ceci gêne l'évacuation de la population, et doit être pris en
compte dans la modélisation.

L'objectif d'Escape est de pouvoir valider des plans d'évacuation comme un Plan
Communal de Sauvegarde (PCS). Un PCS est un document décrivant l'organisation à
adopter face aux risques de désastre naturel, industriel ou sanitaire. Il est
écrit par des experts avec l'équipe de la mairie. Il y indique tout ce qu'il
faut faire pour informer la population sur les risques, l'alerter en cas de
crise, la guider lors de l'évacuation, et l'assister si besoin est. Pour
valider un PCS, Escape doit vérifier la faisabilité et l'efficacité des
dispositions proposées dans le plan.

Escape s'intéresse principalement à tester les dispositions prises lors de la
crise. Escape cherche entre autres à estimer le temps qu'il faudrait pour que la
population évacue la ville dans différents scénarios et selon les dispositions
prises dans le PCS. L'équipe de la mairie peut ainsi tester différentes actions,
les évaluer et les comparer, pour faire un meilleur PCS.

Pour simuler l'évacuation de la ville, la méthode mise en œuvre est un système
multi-agents. Un système multi-agents est un programme informatique où l'on
simule plusieurs agents autonomes. Un agent est une entité située dans un
environnement, auquel on associe un comportement le rendant capable de prendre
des décisions et d'interagir avec son environnement. Dans Escape, les agents
sont les civils (piétons ou automobilistes) et les agents de gestion de crise.

Dans les systèmes multi-agents, les agents évoluent sur une structure
informatique modélisant l'espace.\comment{Pas toujours, il existe des SMA sans
espace associé} Il existe plusieurs types de structures pour modéliser l'espace.
En effet, différentes structures de modélisations permettent de simuler les
différentes caractéristiques de l'espace.

Pour que les agents prennent des décisions, nous devons leur fournir des
connaissances et un algorithme qui, avec ces connaissances, indique la décision
à prendre. Nous nous intéressons ici à la structure modélisant la connaissance
de l'espace. On appelle connaissance de l'espace toutes les informations
relatives à l'environnement qu'a un individu à sa disposition. Il y a une grande
diversité de connaissance, et des individus avec des connaissances différentes
font des choix différents. Ainsi la modélisation se doit de pouvoir représenter
cette diversité.

Dans la section "\nameref{etat}" nous verrons les différentes structures
utilisées dans la littérature pour modéliser l'espace et pour modéliser la
connaissance de l'espace.

On fera bien attention à différencier l'espace réel, la modélisation
informatique de cet espace, la connaissance qu'ont les individus de l'espace, et
la modélisation de cette connaissance. Aussi, on parlera «~d'individus~» lorsque
l'on fera référence aux entités réelles et «~d'agents~» pour désigner leur
modélisation informatique (résumé dans la table \ref{tab:vocab}).

\begin{table}
  \noindent\makebox[\textwidth]{
  \begin{tabular}{|c||c|c|c|}
    % \hline
    % {}&Espace objectif&Espace subjectif&Entités \\
    \hline
    Environnement réel & Espace & Connaissance de l'espace & Individu \\
    \hline
    Modélisation informatique & Modélisation de l'espace & Modélisation de la
    connaissance &
    Agents\\
    \hline
  \end{tabular}
  }
  \caption{Le vocabulaire utilisé, en fonction de si on parle de
  l'environnement réel ou de la modélisation informatique}
  \label{tab:vocab}
\end{table}

Notre objectif est de modéliser la réalité le plus fidèlement possible. Pour
cela il faut bien comprendre ce que l'on veut modéliser. Dans la section qui
suit nous allons décrire les caractéristiques de l'espace et de la connaissance
de cet espace que nous voulons modéliser.

\subsection{L'espace et la connaissance de l'espace}

On définit l'espace comme étant l'étendue de terre où les individus peuvent se
déplacer. Dans le contexte de l'évacuation de ville, il est plus pertinent de
définir l'espace de la ville comme l'ensemble des modes de transports
disponibles dans la ville. C'est-à-dire le réseau pédestre, le réseau routier,
les transports en commun, etc. En effet, c'est principalement via ces réseaux
que les individus vont évacuer la ville.

La connaissance de l'espace est l'ensemble des informations qu'a un individu à
propos des modes de transport disponibles dans sa ville et leurs modalités. On y
inclut l'ensemble des rues qu'il connaît, les distances, les routes, les lignes
des transports en commun, leurs horaires, etc.

Les habitants d'une ville n'ont pas la même connaissance de l'espace. Par
exemple, certaines personnes connaissent des rues ou des raccourcis que d'autres
ne connaissent pas, comme illustré sur la figure~\ref{fig:espace}. Autrement
dit, les individus ont une connaissance personnelle de l'espace. Cette
connaissance peut être partielle et/ou inexacte. On peut supposer qu'il existe
autant de connaissances de l'espace possibles que d'individus. Cependant cette
diversité des connaissances est contrebalancée par le fait que les individus ont
accès à des sources d'information communes comme les cartes, les panneaux
indicateurs dans les rues, et les connaissances des autres individus lorsqu'ils
discutent.

\tikzset{legende/.style={draw, rectangle, rounded corners, fill=blue!20}}
\begin{figure}
  \centering
  \begin{tikzpicture}[scale=2]
    \newcommand{\drawMissing}[3]{
      \draw[dotted, thin] (#1) -- (#2) node[midway, #3] {?};
    }

    \newcommand{\drawSpace}[1]{
      \node[draw, circle] (C) at (0,2) {Maison 1};
      \node[draw, circle] (D) at (0.5,0) {Maison 2};
      \node[draw, circle] (A) at (1,1) {Travail};
      \node[draw, circle] (B) at (0,1) {};
      \node[legende] at (0.5, -0.75) {#1};
    }

    \begin{scope}[shift={(0,0)}]
      \drawSpace{Espace objectif}
      \draw (A) to (C);
      \draw (B) to (C);
      \draw (B) to (D);
      \draw (A) to (D);
    \end{scope}

    \begin{scope}[shift={(-1.5,-4)}]
      \drawSpace{Connaissance de l'individu 1}
      \draw (A) to (C);
      \draw (B) to (C);
      \draw (B) to (D);
      \drawMissing{A}{D}{right}
    \end{scope}

    \begin{scope}[shift={(1.5,-4)}]
      \drawSpace{Connaissance de l'individu 2}
      \drawMissing{A}{C}{above right}
      \draw (B) to (C);
      \draw (B) to (D);
      \draw (A) to (D);
    \end{scope}

  \end{tikzpicture}
  \caption{La différence entre l'espace objectif, et la connaissance de
  l'espace. L'individu 1 ne connait pas le chemin entre la maison 2 et le lieu
  de travail, et réciproquement pour l'individu 2}
  \label{fig:espace}
\end{figure}

Souvent dans une ville plusieurs réseaux de transports coexistent, par exemple
le réseau routier et le réseau de transports en commun. Ces réseaux ne sont pas
forcément accessibles par tous les individus, et certains individus peuvent
emprunter plusieurs réseaux consécutivement. Par exemple, un piéton peut
emprunter les transports en commun puis décider de marcher jusqu'à sa
destination, tandis qu'un automobiliste est obligé d'emprunter le réseau routier
ou bien de garer sa voiture pour devenir un piéton.

Ces types de déplacements multimodaux apportent une caractérisation des
individus supplémentaire. Non seulement les individus ne connaissent pas la
ville de la même façon, mais leur accès aux moyens de transports de la ville est
différent.

De plus, les différents réseaux peuvent être affectés différemment par les
aléas. Une rue peut, par exemple, ne pas être praticable pour les automobilistes
à cause de la pluie, mais être traversable par les piétons.

Finalement, il y a deux concepts à modéliser~: l'espace, autrement dit
l'ensemble des modes de transports de la ville~; et la connaissance de cet
espace, c'est-à-dire toutes les informations qu'ont les individus sur cet
espace. Cependant, lors d'une crise naturelle ou industrielle, l'espace est
modifié par les aléas et les individus prennent progressivement connaissance
de la modification de cet espace, donc leurs connaissances sont modifiées. Les
modifications de l'espace et des connaissances sont dynamiques d'où la nécessité
de s'intéresser à l'évolution temporelle de l'espace et des connaissances.

  \subsection{Évolution temporelle}

L'espace de la ville peut être amené à se modifier au cours de la crise. Une
modification de cet espace est tout changement affectant le déplacement des
individus. Les modifications peuvent prendre plusieurs formes. Il y a des
modifications qui empêche complètement l'accès à certaines zones, e.g., la
propagation d'un gaz toxique, dû à l'explosion d'une usine. D'autre qui coupe
une arête d'un réseau de transport, e.g., l'effondrement d'un pont. Il y a aussi
des modifications plus légères qui n'empêchent pas totalement l'accès à une
zone, mais le rends plus difficile, e.g., le terrain devient boueux.

La modification peut avoir plusieurs causes, le cas le plus courant dans notre
cas étant que l'aléa altère une partie du réseau. La modification de l'espace
peut aussi être dus aux actions de l'homme. Typiquement, un PCS peut
recommander de fermer un pont par soucis de sécurité.

Lorsque l'espace est modifié pendant la crise, les individus ne sont à priori
pas immédiatement au courant de cette modification. Pour l'intégrer à leur
connaissance, il faut soit qu'ils la voient, soit que l'information leur
parvienne par une source quelconque.

La multitude des sources d'information disponibles et leur non-harmonisation
provoque des conflits. En fait les sources d'information ne se mettent pas
toutes à jour automatiquement et les individus n'ont pas accès à toutes les
sources d'informations en permanence, cela provoque des situations où certains
individus reçoivent des informations contradictoires de plusieurs sources
différentes. Dans une telle situation, les individus doivent réfléchir pour
choisir quelle information ils doivent conserver. Ils peuvent juger quelle
information a le plus de chance d'être véridique, en prenant en compte la date
de dernière mise à jour de la source ou la confiance qu'ils accordent à cette
source.

Parfois des individus n'ayant pas les mêmes connaissances se regroupent. Comme
leurs décisions sont prises à partir des informations qu'ils possèdent, leur
différence de connaissances entraine une différence de décisions. Cependant,
comme ils sont dans le même groupe, ils doivent se mettre d'accord pour prendre
la même décision. De fait, l'un des individus doit convaincre les autres que sa
décision est la meilleure car sa connaissance est plus proche de la réalité.

Ces évolutions temporelles amènent les individus à devoir sans cesse réévaluer
leurs choix, puisqu'un chemin autrefois jugé optimal doit être revu s'il est
affecté par l'aléa. De plus les individus peuvent apprendre de nouvelles
informations, ce qui les obligent à réévaluer leurs choix à la lumière de
celles-ci. Pour modéliser les prises de décisions des individus, on doit faire
un certain nombre de calculs. La fréquence des modifications de l'espace et des
connaissances de l'espace pose la question de la complexité à prendre en compte
pour résoudre le problème.

  \subsection{Contraintes techniques}

La ville est un espace grand et complexe, et le nombre d'agents est gigantesque.
Pour se donner un ordre d'idée, à Rouen on compte 110 169 habitants sur une
superficie de 21 km$^2$. De fait il faut faire très attention à la taille
mémoire que va prendre la modélisation.\comment{On veut stocker un graphe avec
toutes les rues et les intersections}

De plus, les agents appliquent souvent des algorithmes particulièrement long sur
des structures de données complexe tel que la recherche de plus court chemins.
Il faut veiller à ce que la modélisation choisie permette de faire tourner ces
algorithmes en un temps raisonnable.

  \subsection{Conclusion}

Escape est une application faite pour simuler des évacuations de ville à l'aide
d'un système multi agents. On veut modéliser l'espace et l'ensemble des
informations que peut avoir un individu à propos de cet espace. Dans le but de
faire une modélisation fidèle nous avons décrit les caractéristiques de l'espace
que nous voulons conserver. Nous avons évoqué la complexité des réseaux, des
déplacements multimodaux, les différences d'informations qu'ont les individus,
et de l'évolution temporelle de l'espace et des connaissances. Un point à ne pas
négliger pour produire une modélisation fidèle est la complexité du problème. En
effet, nos capacités de calculs et d'espace mémoire ne sont pas infinies. Dans la
prochaine section nous allons faire un état de l'art des structures
informatiques pour modéliser l'espace et la connaissance de l'espace.

\section{État de l'art}
\label{etat}

Il existe deux catégories de modélisations de l'espace les modélisations en
matrice (i.e. raster) et les modélisations objets (i.e. vecteur). La
modélisation en matrice est utilisé notamment dans \cite{batty1999} où l'on
découpe l'espace entier en portions, et chaque portion de l'espace est enrichie
d'informations. Les modélisations en matrices sont principalement utilisées pour
modéliser les étendues, comme les habitations, les lacs, les reliefs... Par
opposition, les modélisations objets où l'on considère les objets (e.g., les
routes, immeubles, individus) qui constituent l'espace, sont utilisées pour
modéliser les relations entre les objets dans l'espace.

Dans ces modélisations objets de l'espace on veut modéliser les relations
entre les objets. Le concept de Region Connection Calculus
(RCC-8) \cite{randell1992} donne les bases formelles des définitions
des relations topologiques (se touchent, se superposent, à l'intérieur
de...) et peut nous être utile.

Pour les modélisations objets on utilise souvent les graphes, où les objets sont
des nœuds et les relations entre ces objets sont représentés par des arêtes. Il y
a pleins de façon d'utiliser les graphes pour modéliser l'espace, comme le
montre \cite{dale2010}. Un autre exemple de modélisation de l'espace par des
graphes sont les \emph{maptree} proposés par \cite{worboys2012}. Ils se
focalisent sur la modélisation des frontières.

La façon classique de modéliser le réseau de rues est de considérer que les
intersections des rues sont des nœuds et relier chaque nœud s'il y a un tronçon
de rue entre les deux. Comme le montre \cite{porta2005}, on peut aussi prendre
le graphe dual et avoir chaque tronçon de rue comme nœud et deux nœuds sont
reliés si les deux tronçons se croisent. Ou bien encore chaque rue est un nœud
et relier deux nœuds si les rues s'intersectent, voir la figure \ref{fig:dual}.
On peut faire de même pour le réseau routier ou celui des transports en commun.

\begin{figure}
  \centering
  \begin{tikzpicture}[scale=1]
    \begin{scope}
      \node[draw, circle] (A) at (0,4) {A};
      \node[draw, circle] (B) at (0,2) {B};
      \node[draw, circle] (C) at (0,0) {C};
      \node[draw, circle] (D) at (2,4) {D};
      \node[draw, circle] (E) at (2,2) {E};
      \node[draw, circle] (F) at (2,0) {F};
      \node[legende] at (1,-1) {Graphe primal};

      \draw (A) -- (B) node[left, midway] {$\alpha$};
      \draw (C) -- (B) node[left, midway] {$\beta$};
      \draw (E) -- (B) node[above, midway] {$\gamma$};
      \draw (D) -- (E) node[right, midway] {$\delta$};
      \draw (F) -- (E) node[right, midway] {$\epsilon$};
    \end{scope}

    \begin{scope}[shift={(-3,-5)}]
      \node[draw, circle] (al) at (-0.5,3) {$\alpha$};
      \node[draw, circle] (be) at (-0.5,1) {$\beta$};
      \node[draw, circle] (ga) at (1,2) {$\gamma$};
      \node[draw, circle] (de) at (2.5,3) {$\delta$};
      \node[draw, circle] (ep) at (2.5,1) {$\epsilon$};
      \node[legende] at (1,0) {Graphe dual};

      \draw (al) -- (be) node[left, midway] {B};
      \draw (al) -- (ga) node[above right, midway] {B};
      \draw (ga) -- (be) node[below right, midway] {B};
      \draw (ga) -- (de) node[above left, midway] {E};
      \draw (ga) -- (ep) node[below left, midway] {E};
      \draw (de) -- (ep) node[right, midway] {E};
    \end{scope}

    \begin{scope}[shift={(3, -3)}]
      \node[draw, circle] (ab) at (-0.5,0) {$\alpha\beta$};
      \node[draw, circle] (ga) at (1,0) {$\gamma$};
      \node[draw, circle] (de) at (2.5,0) {$\delta\epsilon$};
      \node[legende] at (1,-2) {Graphe des rues};

      \draw (ab) -- (ga) node[above, midway] {B};
      \draw (de) -- (ga) node[above, midway] {E};
    \end{scope}
  \end{tikzpicture}
  \caption{Comparaisons des graphes primal, dual et des rues, du même réseau des
  rues. Les lettres grecs représentent des tronçons de rues, et les lettres
  latines représentent les intersections.}
  \label{fig:dual}
\end{figure}

Notre espace a beaucoup de détails, il peut être intéressant de réduire le
niveau de détails d'une modélisation pour réduire la complexité des algorithmes.
On pense par exemple à faire des calculs de plus de court chemins sur une carte
générale, pour avoir une idée du chemin global à prendre. La granularité,
c'est-à-dire la modélisation de l'espace selon plusieurs degrés de précisions,
est l'outil idéal pour diminuer la complexité du graphe sans perdre
d'informations. On donne un exemple de réseau à plusieurs niveaux de précisions
sur la figure \ref{fig:generalisation}. \cite{stell1998} donne un outil formel
d'organisation des modélisations sur plusieurs degrés de précisions sémantiques
et géographiques, les \emph{stratified map space}.

\begin{figure}
  \tikzset{place/.style={draw, circle, fill=white}}
  \tikzset{bigplace/.style={draw, circle, dashed, fill=gray!20}}
  \begin{tikzpicture}
    \node[bigplace, minimum width=4cm] (bA) at (1,7) {};
    \draw (bA) node[above] {\Huge $\alpha$};
    \node[bigplace, minimum width=4cm] (bD) at (6,4) {};
    \draw (bD) ++ (0,0.25) node[] {\Huge $\beta$};
    \node[bigplace, minimum width=3cm] (bG) at (2,2) {};
    \draw (bG) ++ (0,0.75) node {\Huge $\gamma$};
    \node[bigplace, minimum width=3cm] (bI) at (10,2) {};
    \draw (bI) ++ (0,0.85) node[] {\Huge $\delta$};

    \node[place] (A) at (0,8) {A};
    \node[place] (B) at (2,8) {B};
    \node[place] (C) at (1,6) {C};
    \node[place] (D) at (5,5) {D};
    \node[place] (E) at (7,5) {E};
    \node[place] (F) at (6,3) {F};
    \node[place] (G) at (1,2) {G};
    \node[place] (H) at (3,2) {H};
    \node[place] (I) at (10,2) {I};

    \draw (A) -- (B) (C) -- (B) (A) -- (C);
    \draw (D) -- (E) (E) -- (F) (D) -- (F);
    \draw (G) -- (H);

    \draw (B) -- (D) (C) -- (D) (C) -- (G) (H) -- (F) (F) -- (I) (E) -- (I);

    \draw[dashed] (bA) -- (bD) (bA) -- (bG) (bG) -- (bD) (bD) -- (bI);
  \end{tikzpicture}
  \caption{Un réseau avec deux niveaux de précision.}
  \label{fig:generalisation}
\end{figure}

Maintenant que nous avons évoqué différentes modélisations de l'espace, nous
nous intéressons à la façon d'intégrer les différentes modalités de transports à
la modélisation. \cite{jguirim2015} propose d'associer à chaque agent un profil
indiquant les modalités de transports qu'ils peuvent emprunter. Cette solution
nous semble adéquate, puisqu'elle permet de modéliser les comportements qu'on
observe. Elle a aussi l'avantage de ne pas être complexe, ni d'exiger une grande
place en mémoire.\comment{Il faudrait argumenter ce point}

En ce qui concerne la modélisation des connaissances, il existe des études sur
les niveaux de connaissances, comme dans \cite{stern1988} qui fait une étude
pour mesurer le niveau de connaissance de la ville qu'ont les conducteurs
professionnels par rapport aux conducteurs amateurs. Il y a aussi
\cite{quinn2018} qui s'intéresse au rapport qu'ont les gens aux risques naturels
près de chez eux, en fonction de leur attachement à leur quartier. Cela ne
correspond pas à ce qu'on veut faire, ce sont des études sur les comportements
des individus plutôt que sur la modélisation de ces comportement. Cependant ils
pourraient être utilisés pour parfaire notre modèle des individus.
\cite{kuipers1978} a une approche plus informatique, et propose un modèle
reproduisant la façon dont les individus se représentent l'espace. L'idée étant
que les gens retiennent surtout les chemins entre leurs points d'intérêts et la
position relative de certains lieux par rapport à des points de références (un
monument visible de loin par exemple).

Pour l'instant nous nous sommes penchés sur les modélisations de la connaissance
d'un individu seul. Cependant dans notre simulation on veut modéliser la
connaissance de plusieurs individus. Malheureusement, il semble qu'il y a peu
d'articles détaillant la façon dont on peut modéliser la diversité des
connaissances de l'espace qu'ont les individus. Comme les individus peuvent
avoir une partie de connaissance en commun il faudrait pouvoir la modéliser.
Idéalement on aimerait une façon de modéliser la différence entre le graphe
représentant l'espace et celui représentant la connaissance qu'a un individu de
cet espace.

La connaissance de l'espace est amenée à évoluer dans le temps, nous allons voir
comment la dimension temporelle peut s'introduire dans notre modèle. Malgré les
différentes façons de décrire le temps, une semble faire consensus, celle
développée par \cite{allen1985}. Il y théorise le principe d'instants. Les
instants temporels étant ce qui constitue le début et la fin des intervalles. Il
définit une algèbre avec les intervalles de temps comme objet de base et avec
ceux-ci définit des relations et tout ce qu'il faut pour avoir une algèbre
temporelle complète.

Lorsqu'on parle de dynamique spatio-temporelle on voit souvent des articles qui
s'intéressent aux processus modifiant l'espace. Certains, comme\linebreak
\cite{claramunt1995} décrivent les processus et évènements comme des entités. Ce
sont des \emph{Temporal Geographic Information Systems}(T-GIS),
\cite{siabato2018} fait une étude de tous travaux fait sur les T-GIS. D'autres
comme \cite{delmondo2011} et \cite{costes2015} se penchent sur la représentation
des relations entre les objets dans des espaces et des instants potentiellement
différents.

\comment{Il faut reformuler cette transition}
Mais ces articles ne parlent pas de la complexités en place et en temps des
modélisations qu'ils proposent. On se tourne alors vers les \emph{multi level
overlay graphs} de \cite{holzer2009}. Le but de ces constructions est de
diminuer la complexité des algorithmes du plus court chemin sur un graphe en
faisant une partie des calculs en avance. L'idée est de sélectionner les nœuds
les plus importants pour le calcul de chemins, puis on relie ces nœuds avec une
méta-arète, représentant le plus court chemin entre les nœuds. Cela permet
ensuite de retrouver le plus court chemin entre tous les nœuds plus rapidement.
Comme la simulation multi-agents va souvent demander aux agents de décider où
aller, refaire tout le calcul de plus court chemin à chaque fois va demander
beaucoup de puissance de calcul. Dans notre cas, le modèle est voué à évoluer au
cours de la simulation, de fait \cite{bruera2008} qui rend ces \emph{overlay
graphs} dynamiques, est particulièrement intéressant.

On a vu plusieurs outils pour modéliser l'espace RCC-8 pour formaliser les
relations topologiques, les graphes pour modéliser les relations entre les
objets. On a aussi vu des modélisations de l'espace les maptree, les graphes
duaux. Les modélisations qui permettent de réduire la complexité de l'algorithme
du plus court chemin paraissent les plus pertinentes, c'est-à-dire les
\emph{multi level overlay} ainsi que les graphes duaux. Cependant nous n'avons
pas trouvé de littérature traitant de la façon de modéliser les différences des
connaissances des individus. Cela constitue notre problématique majeur.

\bibliographystyle{apalike} \bibliography{biblio.bib}

\end{document}
