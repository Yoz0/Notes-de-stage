\documentclass[a4paper]{article}

\usepackage[utf8]{inputenc}
\usepackage[T1]{fontenc}
\usepackage{lmodern}
\usepackage[french]{babel}
\usepackage{xcolor}
\usepackage{tikz}
\usepackage{nameref}
\usepackage{comments}
\usepackage{amsmath}

\title{Propositions de modélisations}

\author{Stéphane Kastenbaum}

\date{2018-08-20}

\begin{document}

\maketitle

\section{Introduction}

Blablabla Escape, ça suit la synthèse. On va proposer des solutions, le but est
de trouver la meilleure. En fait on propose une "solution naïve" qui nous sert
de point de départ auquel on va proposer des améliorations. Les améliorations ne
sont pas complètement développé le but est de montrer l'idée. Ça se trouve les
améliorations ne sont pas pertinentes, ce ne sont que des propositions.

\section{Solution naive}

  \subsection{Espace}

Dans cette solution on représente l'espace avec plusieurs graphes, chacun étant
un moyen de transport. Cela permet de montrer les modalités de transports
différents. Les nœuds sont les endroits où les agents peuvent se
situer, les arêtes sont les chemins qu'ils peuvent emprunter. Par exemple, sur
la figure \ref{fig:multimodal}, un agent peut aller du nœud B au nœud D en
empruntant le réseau piéton (en noir), ou bien en prenant le bus (en rouge),
mais pas en prenant la voiture (en bleu).

\begin{figure}
  \tikzset{place/.style={draw, minimum width=2em, circle, fill=white}}
  \tikzset{rplace/.style={draw=red, minimum width=2.5em, circle}}
  \tikzset{bplace/.style={draw=blue, minimum width=2.25em, circle}}
  \begin{tikzpicture}[scale=0.75]
    \node[place] (A) at (0,8) {A};
    \node[place] (B) at (2,8) {B};
    \node[place] (C) at (1,6) {C};
    \node[place] (D) at (5,5) {D};
    \node[place] (E) at (7,5) {E};
    \node[place] (F) at (6,3) {F};
    \node[place] (G) at (1,2) {G};
    \node[place] (H) at (3,2) {H};
    \node[place] (I) at (10,2) {I};

    \node[rplace] (rA) at (A) {};
    \node[rplace] (rB) at (B) {};
    \node[rplace] (rD) at (D) {};
    \node[rplace] (rI) at (I) {};

    \node[bplace] (bC) at (C) {};
    \node[bplace] (bD) at (D) {};
    \node[bplace] (bE) at (E) {};
    \node[bplace] (bF) at (F) {};
    \node[bplace] (bG) at (G) {};
    \node[bplace] (bI) at (I) {};

    \draw (A) -- (B) -- (C) -- (A) ;
    \draw (D) -- (E) -- (F) -- (D);
    \draw (G) -- (H);
    \draw (B) -- (D) -- (C) -- (G) (H) -- (F) -- (I) -- (E);
    \draw[red, bend left] (rA) to (rB) (rB) to (rD);
    \draw[red] (rD) to (rI);
    \draw[blue, bend left] (bD) to (bE) (bE) to (bI);
    \draw[blue, bend right] (bD) to (bF) (bF) to (bE);
    \draw[blue, bend left] (bG) to (bF);
    \draw[blue, bend right] (bC) to (bD);

  \end{tikzpicture}
  \caption{Un graphe mutli-modal, noir : réseau piéton ; rouge : bus ; bleu :
  réseau routier}
  \label{fig:multimodal}
  \centering
\end{figure}

  \subsection{Connaissance de l'espace}

Ici, chaque agent a sa propre copie de l'espace en mémoire. La copie est
l'espace tel que connais l'agent, elle peut être partielle, inexacte et même
fausse.

Lors de la mise à jour de la connaissance de l'agent, il peut être utile de
savoir d'où viennent les informations que l'agent possède déjà, pour pouvoir les
comparer avec les nouvelles informations. On associe à chaque nœud et arête du
graphe la source d'où vient cette information, ainsi que la date à l'information
a été mise à jour.

  \subsection{Commentaire}

Cette proposition est la première qui vient à l'esprit lorsque le problème est
posé. Elle a l'avantage d'être très fidèle à la réalité, et permet une
modélisation de beaucoup de situations (elle ne fait pas de suppositions).
Malheureusement elle n'est pas utilisable dans ce projet. En effet le nombre de
calcul requis par cette modélisation est énorme. Le but de ce papier est de
proposer différentes améliorations possibles pour rendre cette solution
utilisable, en diminuant le nombre de calculs requis.

Précisément, il y a deux problèmes à cette solution :

\begin{itemize}
  \item

La taille d'un graphe de connaissance est environ égale à la taille du
graphe de l'espace. De sorte que, la place mémoire nécessaire pour stocker
tous les graphes de connaissances est environ égale à \[\text{Nb d'agents} *
\text{Taille du graphe}\]

  \item

La simulation doit faire un calcul de plus court chemins pour chaque
agent. En effet comme chaque agent à sa propre connaissance, lorsque l'agent
cherche comment aller de sa position à une autre position, il doit faire un
calcul de plus court chemins. De plus lorsque sa connaissance de l'espace
est modifié, il doit refaire le calcul car le résultat sera peut-être
différent.

\end{itemize}

\section{Graphes duaux}

On remplace les graphes par leurs duaux ou par les graphes des rues comme fait
\cite{porta2005}.
\comment{Illustration ? La mienne ou celle de porta ?}
  \subsection{Commentaire}

Le but de cette amélioration est de diminuer le nombre d'arêtes et de nœuds dans
les graphes. Le souci est que les algorithmes que l'ont veut utiliser sur ces
graphes fonctionnent différemment avec les graphes des rues. Il faut donc
réadapter les algorithmes, on n'est pas sur que le gain de calculs est
suffisant, voir même significatif.

\section{Stockage de la différence des connaissances}

Au lieu de conserver la réseau complet de la connaissance de l'espace pour
chaque agent, on calcule la différence entre sa connaissance et l'espace, et
on ne stocke que les nœuds et arêtes différents.

Le point important est de trouver la façon la plus optimale pour calculer la
différence entre deux graphes, et pour reconstruire le graphe de connaissance.

\section{Mise en commun de la connaissance}

On définit un petit nombre de différentes connaissance de l'espace, on les
stockent. Pour chaque agent on indique quelle connaissance il possède.

\section{Granularité}

On peut définir des quartiers, et faire les calculs de plus court chemins la
dessus comme dans les overlays graphs voir \cite{holzer2009}.

Le but c'est de diminuer les calculs à faire, en découpant le graphe en quartier
et calculer d'abords les plus courts chemins. Il faut aussi calculer le chemin
entre quartiers.

J'ai du mal à voir comment ça va nous aider, mais j'ai le sentiment que ça peut
effectivement nous aider.

\bibliographystyle{apalike} \bibliography{biblio.bib}

\end{document}
