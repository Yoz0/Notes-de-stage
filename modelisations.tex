\documentclass[a4paper]{article}

\usepackage[utf8]{inputenc}
\usepackage[T1]{fontenc}
\usepackage{lmodern}
\usepackage[french]{babel}
\usepackage{xcolor}
\usepackage{tikz}
\usepackage{nameref}
\usepackage{amsmath}
\usepackage{amssymb}
\usepackage{braket}
\usepackage[colorlinks]{hyperref}
\usepackage[colorinlistoftodos]{todonotes}
\usepackage{comments}
% \newcommand{\comment}[1]{}
\usepackage{algorithm}
\usepackage{algorithmicx}
\usepackage[]{algpseudocode}
\usepackage{glossaries}

\usetikzlibrary{arrows,decorations.markings}

\tikzset{place/.style={draw, minimum width=2em, circle, fill=white}}
\tikzset{rplace/.style={draw=red, minimum width=2.5em, circle}}
\tikzset{bplace/.style={draw=blue, minimum width=2.25em, circle}}
\tikzset{bigplace/.style={draw, circle, dashed, fill=gray!20}}
\tikzset{connaissance/.style={draw, rectangle, rounded corners, fill=yellow!30,
text width=3cm, align=center}}
\tikzset{people/.style={draw, minimum width=2em, circle, fill=blue!20}}
\tikzset{legende/.style={draw, rectangle, rounded corners, fill=blue!20}}
\tikzset{bigarrow/.style={decoration={markings,mark=at position 1 with
    {\arrow[scale=2,>=latex]{>}}},postaction={decorate}}}

\DeclareMathOperator{\poids}{GED}
\DeclareMathOperator{\diff}{diff}
\DeclareMathOperator{\dist}{distance}
\DeclareMathOperator{\connait}{récupererConn}
\DeclareMathOperator{\applique}{appliqueDiff}
\DeclareMathOperator{\parent}{parent}
\DeclareMathOperator{\enfants}{enfants}
\DeclareMathOperator{\majc}{changerConn}
\newcommand{\N}{\mathbb{N}}

\makenoidxglossaries
\newglossaryentry{G}{name={$G(C, E)$},description={Graphe des versions de
connaissances, non-orienté, pondéré par la fonction $\poids$}}
\newglossaryentry{C}{name={$C$},description={$\set{G_k(V_k, E_k) | \forall k}$ :
L'ensemble des versions de connaissances.}}
\newglossaryentry{E}{name={$E$},description={$C \times C$ : Dans le graphe des
connaissances, tous les nœuds sont connectés.}}
\newglossaryentry{F}{name={$F(C,E_F)$}, description={Arbre des versions de
connaissance avec $\emptyset$ comme racines, non-orienté.}}
\newglossaryentry{F'}{name={$F'(C',E_F')$},description={Arbre des
variations de connaissance $\emptyset$ est la racine.}}
\newglossaryentry{O}{name={$O$},description={L'image de $C \times C$ par la
fonction \gls{diff}. Autrement dit, l'ensemble des opérations pour passer d'un
graphe des connaissances à l'autre}}
\newglossaryentry{Vinf}{name={$V_\infty$},description={L'ensemble des
intersections des rues}}
\newglossaryentry{Einf}{name={$E_\infty$},description={L'ensemble des tronçons
de rues}}
\newglossaryentry{A}{name={$A$},description={L'ensemble des agents}}
\newglossaryentry{T}{name={$T$},description={L'ensemble des instants}}
\newglossaryentry{poids}{name={$\poids$},description={$C \times C \to
\N$ : Fonction qui compte le nombre d'opérations pour passer d'un graphe
à l'autre. Acronyme de \emph{Graphe Edit Distance}.}}
\newglossaryentry{applique}{name={$\applique$},description={$C \times O \to C$
: Applique les modifications sur un graphe et renvoie le nouveau graphe.}}
\newglossaryentry{diff}{name={$\diff$},description={$C \times C \to O$ Calcul
les opérations pour passer d'une connaissance à l'autre.}}
\newglossaryentry{connait}{name={$\connait$},description={$A \times T \to C$ :
Une fonction qui pour un agent et un instant lui associe une connaissance.}}
\newglossaryentry{majc}{name={$\majc$},description={$A \times T \times C \to
\emptyset$ La fonction qu'on appelle lorsque l'agent met à jour sa
connaissance.}}
\newglossaryentry{enfants}{name={$\enfants$},description={$C \to P(C)$ :
Fonction qui renvoie les enfants d'un nœud dans un arbre.}}

\newacronym{avec}{AVEC}{arbre des versions de connaissances}
\newacronym{avac}{AVAC}{arbre des variations de connaissances}

\title{Propositions de modélisations}

\author{Stéphane Kastenbaum}

\date{\today}

\begin{document}

\maketitle

\section{Introduction}

Après avoir fait la synthèse de la littérature utile pour notre problème, nous
allons nous pencher sur les solutions envisageables. Le but est de trouver celle
qui répond le mieux à nos attentes (i.e., qui demande le moins de calculs et de
place mémoire). Pour rappel le problème est le suivant~: on veut construire un
modèle de l'espace, et de la connaissance qu'ont des agents de cet espace. On a
des agents qui se déplacent sur une structure. Il y a donc deux structures à
modéliser~: l'espace qui est unique et relativement fixe au cours de la
simulation~; et la connaissance de l'espace qui est différente pour chaque
agent (ou presque) qui évolue différemment pour chaque agent.

Dans la suite on va proposer une "solution naïve" qui nous sert de point de
départ auquel on va ajouter des améliorations. Les améliorations ne sont pas
complètement développées, le but est de montrer l'idée.

\section{Solution naïve}
\label{sec:naive}

Cette proposition est la première qui vient à l'esprit lorsque le problème est
posé. Elle a l'avantage d'être très fidèle à la réalité, et permet une
modélisation de beaucoup de situations (elle ne fait pas de suppositions).


  \subsection{Espace}

Dans cette solution on représente l'espace avec plusieurs graphes, chacun étant
un moyen de transport. Cela permet de définir des modalités de transports
différents. Les nœuds sont les endroits où les agents peuvent se situer, les
arêtes sont les chemins qu'ils peuvent emprunter. Par exemple, sur la figure
\ref{fig:multimodal}, un agent peut aller du nœud B au nœud D en empruntant le
réseau piéton (en noir), ou bien en prenant le bus (en rouge), mais pas en
prenant la voiture (en bleu).

\begin{figure}%{{{ multimodal
  \begin{tikzpicture}
    \node[place] (A) at (0,8) {A};
    \node[place] (B) at (2,8) {B};
    \node[place] (C) at (1,6) {C};
    \node[place] (D) at (5,5) {D};
    \node[place] (E) at (7,5) {E};
    \node[place] (F) at (6,3) {F};
    \node[place] (G) at (1,2) {G};
    \node[place] (H) at (3,2) {H};
    \node[place] (I) at (10,2) {I};
    \node[draw, circle, fill=red] (rouge) at (8,8) { };
    \draw (rouge) node[right=5pt] {Ligne de bus};

    \node[draw, circle, fill=blue] (bleu) at (8,7) { };
    \draw (bleu) node[right=5pt] {Réseau routier};

    \node[draw, circle] (noir) at (8,6) { };
    \draw (noir) node[right=5pt] {Réseau piéton};

    \draw (A) -- (B) -- (C) -- (A);
    \draw (D) -- (E) -- (F) -- (D);
    \draw (G) -- (H);
    \draw (B) -- (D) -- (C) -- (G) (H) -- (F) -- (I) -- (E);
    \draw[red, bend left] (A) to (B) (B) to (D);
    \draw[red] (D) to (I);
    \draw[blue, bend left] (D) to (E) (E) to (I);
    \draw[blue, bend right] (D) to (F) (F) to (E);
    \draw[blue, bend left] (G) to (F);
    \draw[blue, bend right] (C) to (D);
  \end{tikzpicture}
  \caption{Un exemple de graphe mutli-modal}
  \label{fig:multimodal}
  \centering
\end{figure}%}}}

  \subsection{Connaissance de l'espace}

Ici, chaque agent a sa propre copie de l'espace en mémoire. La copie est
l'espace tel que le connait l'agent, elle peut être partielle, inexacte et même
fausse. La figure \ref{fig:connaissance} donne un exemple de plusieurs
connaissance du même espace.

De plus, au cours de la simulation l'agent fait des expériences qui lui
font comprendre que sa connaissance de l'espace n'est pas exacte. Il peut
alors mettre à jour sa connaissance. Une mise à jour de la connaissance peut
affecter toutes les informations qu'a un agent à propos de l'espace. Par
exemple, la mise à jour peut enlever un chemin que l'agent pensait exister,
ou bien en ajouter un que l'agent ne connaissait pas.

Lors de la mise à jour de la connaissance de l'agent, il peut être utile de
savoir d'où viennent les informations que l'agent possède déjà, pour pouvoir les
comparer avec les nouvelles informations. On associe à chaque nœud et arête du
graphe la source d'où vient cette information, ainsi que la date à laquelle
cette information a été mise à jour.

% {{{ fig:connaissance
\newcommand{\ABC}{
  \node[place] (A) at (0,2) {A};
  \node[place] (B) at (2,2) {B};
  \node[place] (C) at (1,0) {C};
}
\newcommand{\espaceReel}{
  \ABC;
  \node[place] (D) at (5,-1) {D};
  \draw (A) -- (C) -- (B) -- (A);
  \draw (C) -- (D) -- (B);
  \node[connaissance]  at (2.5,-2) {Espace réel};
}
\begin{figure}
  \begin{tikzpicture}[scale=1]
    \begin{scope}
      \ABC;
      \draw (A) -- (B) -- (C) -- (A);
      \node[connaissance]  at (1,-1) {Connaissance 1};
    \end{scope}

    \begin{scope}[shift={(4,0)}]
      \ABC;
      \draw (A) -- (B) -- (C);
      \node[connaissance]  at (1,-1) {Connaissance 2};
    \end{scope}

    \begin{scope}[shift={(8,1)}]
      \espaceReel
    \end{scope}
  \end{tikzpicture}
  \caption{Différentes connaissances du même espace}
  \label{fig:connaissance}
  \centering
\end{figure}
% }}}

  \subsection{Limitations}

Cette solution n'est malheureusement pas utilisable dans ce projet. En effet le
nombre de calculs requis par cette modélisation est énorme.
Précisément, il y a deux problèmes à cette solution~:

\begin{itemize}
  \item

La taille d'un graphe de connaissance est environ égale à la taille du
graphe de l'espace. Si bien que la place mémoire nécessaire pour stocker
tous les graphes de connaissance est environ égale à \[\text{Nb d'agents} *
\text{Taille du graphe}\]

  \item

La simulation doit faire un calcul de plus court chemins pour chaque
agent. En effet comme chaque agent à sa propre connaissance, lorsque l'agent
cherche comment aller de sa position à une autre position, il doit faire un
calcul de plus court chemins. De plus lorsque sa connaissance de l'espace
est modifiée, il doit refaire le calcul car le résultat sera peut-être
différent.

\end{itemize}

Le but de ce papier est de proposer différentes améliorations possibles pour
rendre cette solution utilisable en diminuant le nombre de calculs requis. On
va donc explorer des améliorations de cette solution. Il y a deux axes
d'amélioration. Les améliorations sur la modélisation de l'espace, et les
améliorations du stockage de la connaissance.

\section{Améliorations du graphe de l'espace}

Le but de ces améliorations est de travailler directement sur le graphe de
l'espace (et sur les graphes de la connaissance de l'espace) pour optimiser les
calculs de plus court chemins. Les deux propositions essayent de simplifier le
graphe. En effet, on comprend intuitivement que si le graphe est plus simple
(i.e., il a moins de nœuds et d'arêtes), alors l'algorithme sera plus rapide à
exécuter.

  \subsection{Graphes des rues}

    \subsubsection{Principe}

On remplace les graphes par le graphe des rues comme fait \cite{porta2005}. Un
graphe des rues est un graphe où chaque rue est un nœud et où deux nœuds sont
reliés par une arêtes si les rues qu'ils représentent s'intersectent.
\cite{porta2005} utilise cette méthode pour étudier les graphes des villes d'un
point de vue théorique, mais cela semble pouvoir être utilisé pour optimiser les
calculs de plus court chemin.

\comment{Expliquer + en détail}

En effet, comme il y a moins de rues blabla...

    \subsubsection{Gain}

Cette amélioration permet de réduire le nombre de nœuds et d'arêtes dans le
graphe, mais elle ne donne pas un graphe équivalent.

Alternativement on peut, calculer le chemin empruntant le moins de rues.
Intuitivement ça semble relié au chemin le plus court même si on imagine
facilement des cas où emprunter plus de rues est plus rapide.

\comment{Cette partie est embrouillée entre les deux alternatives : Adapter
l'algorithme de Dijkstra pour trouver le plus court chemins, ou calculer le
chemins utilisant le moins de rues}

    \subsubsection{Limitations}

Le but de cette amélioration est de diminuer le nombre d'arêtes et de nœuds dans
les graphes. Le souci est que les algorithmes que l'on veut utiliser sur ces
graphes fonctionnent différemment avec les graphes des rues. Il faut donc
réadapter les algorithmes. La réadaptation de l'algorithme peut être assez
complexe, et même rendre le calcul si compliqué que cela ne réduit pas le nombre
de calculs.

  \subsection{Granularité}

    \subsubsection{Principe}

On peut définir des quartiers, et faire les calculs de plus court chemins
là-dessus comme dans les overlays graphs voir \cite{holzer2009}.

Dans cette amélioration on ne modifie pas la façon d'organiser la connaissance
des agents, même si on pourrait avoir une organisation différentes  qui prends
parti du découpage en quartiers. On décide de ne pas le développer ici. On doit
aussi découper la connaissance de l'espace en quartiers et il faut que le
découpage soit le même sur les connaissances de l'espace et l'espace réel, pour
que le calcul soit simplifié.

    \subsubsection{Gain}

Le but c'est de diminuer le nombre de calculs à faire. En découpant le graphe en
quartier et en calculant d'abords les plus courts chemins sur ces quartiers, on
espère diminuer le nombre de calculs à faire pendant la simulation.

    \subsubsection{Limitations}

Ça demande pas mal de calculs préliminaire pour définir les quartiers et
faire les calculs de plus court chemins.

La logique supplémentaire appliqué aux graphes prends de la place mémoire.

\section{Amélioration du stockage de la connaissance}

Ces améliorations se concentrent sur la façon dont on peut modéliser les
différentes connaissances de l'espace possible, le but est de réduire la place
en mémoire prise par la connaissance de l'espace.

Dans ces trois améliorations on ne fait pas de modifications sur le graphe de
l'espace, on considère qu'on modélise l'espace par un graphe tout de même.

  \subsection{Mise en commun de la connaissance}
  \label{sec:miseEnCommun}

    \subsubsection{Principe}

On identifie les différentes connaissances de l'espace, on les stocke. Pour
chaque agent on indique quelle connaissance il possède. On donne un exemple sur
la figure \ref{fig:miseEnCommun}.

\begin{figure}%{{{fig:miseEnCommun
  \begin{tikzpicture}[scale=1]
    \node[connaissance] (A) at (0,0) {Espace réel};
    \node[connaissance] (B) at (5,0) {Connaissance espace selon google maps};
    \node[connaissance] (C) at (10,0) {Connaissance espace version 2};

    \node[people] (p1) at (0,-3) {U};
    \node[people] (p2) at (2,-3) {V};
    \node[people] (p3) at (4,-3) {W};
    \node[people] (p4) at (6,-3) {X};
    \node[people] (p5) at (8,-3) {Y};
    \node[people] (p6) at (10,-3) {Z};

    \draw (p1) to (A);
    \draw (p2) to (C);
    \draw (p3) to (B);
    \draw (p4) to (A);
    \draw (p5) to (C);
    \draw (p6) to (A);
  \end{tikzpicture}
  \caption{Plusieurs individus (les lettres de U à Z) ont les mêmes
  connaissances.}
  \label{fig:miseEnCommun}
\end{figure}%}}}

    \subsubsection{Gain}

En mutualisant les connaissances des agents, le but est de prendre moins de
place en mémoire. Cette solution est beaucoup plus efficace s'il y a peu de
différentes connaissances de l'espace, c'est-à-dire si beaucoup d'agents ont la
même connaissance.

    \subsubsection{Limitations}

Ça résout pas énormément de choses s'il y a beaucoup de connaissances
différentes. Par contre ça n'a pas de vraiment de désavantage.

  \subsection{Stockage de la différence des connaissances}
  \label{sec:difference}

    \subsubsection{Principe}

Au lieu de conserver le réseau complet de la connaissance de l'espace pour
chaque agent, on calcule la différence entre sa connaissance et l'espace réel,
puis on ne stocke que les nœuds et arêtes différents, comme montré sur la figure
\ref{fig:difference}.

\begin{figure}%{{{fig:difference
  \noindent\makebox[\textwidth]{
  \begin{tikzpicture}
    \begin{scope}
      \espaceReel
    \end{scope}

    \begin{scope}[scale=0.5, shift={(11,0)}]
      \draw (0,0) -- (2,0) -- (2,0.5) -- (0,0.5) --(0,0);
    \end{scope}

    \begin{scope}[shift={(8,0)}]
      \ABC;
      \draw (A) -- (B) -- (C) -- (A);
      \node[connaissance]  at (1,-1) {Connaissance 1};
    \end{scope}

    \begin{scope}[scale=0.5, shift={(23,0)}]
      \draw (0,0) -- (3,0) -- (3,0.4) -- (0,0.4) --(0,0);
      \draw (0,0.7) -- (3,0.7) -- (3,1.1) -- (0,1.1) --(0,0.7);
    \end{scope}

    \begin{scope}[shift={(14,0)}]
      \node[place, dotted] (C) at (0,0) {C};
      \node[place, dotted] (B) at (1,2) {B};
      \node[place] (D) at (4,-1) {D};
      \draw (C) -- (D) -- (B);
      \node[connaissance] at (2,-2) {Différence à retenir};
    \end{scope}
  \end{tikzpicture}
  }
  \centering
  \caption{Exemple de calcul de la différence entre la connaissance et la
  réalité}
  \label{fig:difference}
\end{figure}%}}}

Le point important est de trouver la façon la plus optimale pour calculer la
différence entre deux graphes, et pour reconstruire le graphe de connaissance,
ou pour faire nos calculs de plus court chemins sans reconstruire le graphe.

Sur ce sujet on peut chercher du côté de la \emph{graph edit distance}, qui est
une distance quantitative mesurant la différence entre deux graphes. On veut la
différence qualitative entre les graphes, qui permet de reconstruire un graphe à
partir de l'autre et de la différence.

En fait on compte stocker toutes les opérations nécessaire au passage d'un
graphe à un autre, les opérations qu'on autorise sont :

\begin{itemize}
  \item Ajouter un nœud
  \item Enlever un nœud s'il n'a plus d'arêtes relié à lui
  \item Ajouter une arête entre deux nœuds existants
  \item Enlever une arête entre deux nœuds existants
\end{itemize}

Dans la suite on va considérer que les poids des arêtes sont tous de 1, et
considérer que le coups en mémoire de toutes ces actions est de 1 pour faciliter
l'explication.

    \subsubsection{Gain}

Le but recherché ici est de diminuer la place mémoire prise par le stockage de
la connaissance. On suppose que l'espace et la connaissance qu'a un individu de
cette espace sont assez proche, et donc stocker uniquement la différence est
plus léger que stocker les deux versions.

    \subsubsection{Limitations}

Le calcul initial de la différence peut être lourd, et si la différence est très
grande, cela ne diminue pas beaucoup (voir pas) la place.

Il faut réadapter l'algorithme de Dijkstra mais à vu de nez ça ne doit pas être
trop compliqué.

  \subsection{Arbre des variations de connaissances}

    \subsubsection{Principe}

On va maintenant décrire une structure qui centralise les différentes
connaissances des agents (comme dans la section \ref{sec:miseEnCommun}), tout en
ne stockant que la différence entre les graphes (comme dans la section
\ref{sec:difference}). Cette amélioration est une combination des deux
propositions précédentes, avec des ajouts propres aussi. Le but est de mettre
les différentes connaissances dans une structure faite pour avoir le minimum de
redondance dans notre base de donnés. On appelle cette structure \gls{avac}.

Dans la suite, on appellera \emph{une version de connaissance}, un graphe
représentant l'espace tel que le connait un agent (que l'on appelait
précédemment connaissance de l'espace). Contrairement au monde réel où tous les
individus ont tous une connaissance de l'espace différente, on considère que
dans notre contexte de simulation multi-agents le nombre de versions de
connaissances différentes est relativement bas. Aussi pour une question de
simplification, nous allons considérer qu'on utilise la représentation de
l'espace décrite dans la section \ref{sec:naive} (i.e., les nœuds représentent
des intersections de rues, et les arêtes représentent des tronçons de rues). On
peut cependant réadapter ce qui suit avec d'autres modélisations de l'espace.

Pour construire l'\gls{avac}, on commence par faire un méta graphe où les nœuds
représentent les versions de connaissances, et on pondère les arêtes avec le
nombre d'opérations nécessaires pour passer d'un graphe à l'autre (voir figure
\ref{fig:grapheConnaissance}). Ensuite on cherche l'arbre recouvrant minimal,
qu'on nomme \gls{avec} (voir les arêtes en rouge sur la figure
\ref{fig:grapheConnaissance} et l'arbre reproduit sur la figure
\ref{fig:arbreConnaissance}). Enfin à partir de l'\gls{avec}, on construit
l'\gls{avac}, dans lequel chaque nœud ne stocke que les opérations nécessaires
pour construire la connaissance du nœud, en partant de la connaissance du nœud
parent (voir la figure \ref{fig:arbreModifConnaissance}).

Une fois que l'\gls{avac} est construit on va initialiser nos agents en
indiquant dans chaque agent quelle version de connaissance il possède.

Ensuite, on lance la simulation.\comment{Nul} Si, au cours de la simulation la
connaissance de l'agent évolue, on a deux possibilités:

\begin{itemize}

  \item Ou bien l'agent atteint une connaissance qui est déjà dans l'arbre. Dans
    ce cas on indique juste que l'agent change de version de connaissances (par
    exemple, dans la figure \ref{fig:grapheConnaissance} : si un agent avec la
    connaissance 2 découvre le nœud C et l'arête AC alors on dit qu'il passe à
    la connaissance 3).

  \item Ou bien l'agent atteint une connaissance qui n'est pas dans l'arbre.
    Dans ce cas on crée une nouvelle connaissance qu'on ajoute à l'arbre, puis
    on met à jour la version de connaissance dans l'agent.

\end{itemize}

Avec cette méthode il se peut que certaines connaissances ne soit plus utilisées
au cours de la simulation (si tous les individus rattachés à une connaissance
changent de connaissance). Dans ce cas on peut envisager d'enlever la
connaissance de l'arbre, tout en gardant la structure globale.

\begin{figure}%{{{fig:grapheConnaissance
  \begin{tikzpicture}
    \begin{scope}[shift={(5, 5)}]
      \node[red, bigplace] (es) at (0,0) {\Huge $\emptyset$};
    \end{scope}

    \begin{scope}
      \node[red, bigplace, minimum width=6cm] (k1) at (2,1) { };
      \node[place] (B) at (1,3) {B};
      \node[place] (C) at (0,1) {C};
      \node[place] (D) at (4,0) {D};
      \draw (B) -- (C) -- (D);
      \node[connaissance] at (2,-1) {Connaissance 1};
    \end{scope}

    \begin{scope}[shift={(0,-5)}]
      \node[red, bigplace, minimum width=3cm] (k2) at (1,0) { };
      \node[place] (A) at (0,0) {A};
      \node[place] (B) at (2,0) {B};
      \draw (B) -- (A);
      \node[connaissance] at (1,-1) {Connaissance 2};
    \end{scope}

    \begin{scope}[shift={(9,0)}]
      \node[red, bigplace, minimum width=4cm] (k3) at (1,1) { };
      \ABC
      \draw (B) -- (A) -- (C);
      \node[connaissance] at (1,-1) {Connaissance 3};
    \end{scope}

    \begin{scope}[shift={(5,-10)}]
      \node[red, bigplace, minimum width=7cm] (k4) at (2.5,1.5) { };
      \node[place] (A) at (0,3) {A};
      \node[place] (B) at (2,3) {B};
      \node[place] (D) at (5,0) {D};
      \draw (A) -- (B) -- (D);
      \node[connaissance] at (2,0) {Connaissance 4};
    \end{scope}

    \draw[red] (es) to node[above left] {\Large 5} (k1);
    \draw[bend left, red] (es) to node[right] {\Large 3} (k2);
    \draw (es) to node[above right] {\Large 5} (k3);
    \draw (es) to node[above right] {\Large 5} (k4);

    \draw (k1) to node[left] {\Large 6} (k2);
    \draw (k1) to node[above] {\Large 6} (k3);
    \draw (k1) to node[above right] {\Large 6} (k4);

    \draw[red] (k2) to node[above left] {\Large 2} (k3);
    \draw[red] (k2) to node[above right] {\Large 2} (k4);

    \draw (k3) to node[right] {\Large 4} (k4);
  \end{tikzpicture}
  \centering
  \caption{Graphe répresentant les coûts pour passer d'une connaissance à
  l'autre -- En rouge : l'arbre couvrant minimal}
  \label{fig:grapheConnaissance}
\end{figure}%}}}

\begin{figure}%{{{fig:arbreConnaissance
  \begin{tikzpicture}[scale=3]
    \node[connaissance] (es) at (1,2) {$\emptyset$};
    \node[connaissance] (k1) at (0,1) {Connaissance 1};
    \node[connaissance] (k2) at (2,1) {Connaissance 2};
    \node[connaissance] (k3) at (1,0) {Connaissance 3};
    \node[connaissance] (k4) at (3,0) {Connaissance 4};

    \draw (es) to node[above left] {\Large 5} (k1);
    \draw (es) to node[above right] {\Large 3} (k2);
    \draw (k2) to node[above left] {\Large 2} (k3);
    \draw (k2) to node[above right] {\Large 2} (k4);
  \end{tikzpicture}
  \centering
  \caption{Arbre des versions de connaissance}
  \label{fig:arbreConnaissance}
\end{figure}%}}}

\begin{figure}%{{{fig:arbreModifConnaissance
  \begin{tikzpicture}[scale=3]
    \node[connaissance] (es) at (1,2) {$\emptyset$};
    \node[connaissance] (k1) at (0,1) {$\diff$($\emptyset$, Conn.1)};
    \node[connaissance] (k2) at (2,1) {$\diff$($\emptyset$, Conn.2)};
    \node[connaissance] (k3) at (1,0) {$\diff$(Conn.2, Conn.3)};
    \node[connaissance] (k4) at (3,0) {$\diff$(Conn.2, Conn.4)};

    \draw (es) to (k1);
    \draw (es) to (k2);
    \draw (k2) to (k3);
    \draw (k2) to (k4);
  \end{tikzpicture}
  \centering
  \caption{Arbre des variations de connaissance}
  \label{fig:arbreModifConnaissance}
\end{figure}%}}}

    \subsubsection{Gain}

Cette amélioration diminue la place mémoire prise par la modélisation des
connaissances. En effet, au lieu de stocker toutes les connaissances
individuellement, on ne stocke que le minimum des variations entre elles.

    \subsubsection{Limitations}

Cette solution est difficile à mettre en place, et demande beaucoup de calculs,
c'est un compromis entre quantité de calcul et place mémoire prise.

Il faut faire le calcul des différences entre chaque connaissances, si il y a
beaucoup de connaissances différentes le coût de faire le graphe méta est lourd.
Il y a peut-être moyen de réduire le nombre de calculs en utilisant le fait que
c'est une distance, avec l'inégalité triangulaire.

    \subsubsection{Formalisation}

On va maintenant formaliser la modélisation, pour pouvoir expliciter les
algorithmes qui utiliseront la structure.

\paragraph{Versions de connaissances} Soit l'ensemble des versions de
connaissances, $C = \set{G_k = (V_k, E_k) | V_k \subset V_\infty, E_k \subset
E_\infty, \forall k \in \N}$. Avec \gls{Vinf} l'ensemble des intersections de
rues, et \gls{Einf} l'ensemble des tronçons de rues.

\paragraph{Graphe des versions de connaissances} Soit le graphe des versions
connaissances \gls{G}. \gls{G} est un graphe complet, et les arêtes sont
pondérées par la fonction \gls{poids} (Graphe Edit Distance) qui indique le
nombre d'opération nécessaire pour passer d'une connaissance à l'autre. On donne
un exemple d'un tel graphe sur la figure \ref{fig:grapheConnaissance}.

\paragraph{Arbre des versions de connaissances} Soit \gls{F}, l'arbre couvrant
(i.e., le sous graphe connexe et sans cycles) du graphe \gls{G} minimisant la
fonction \gls{poids}. On donne un exemple d'\gls{avec} sur la figure
\ref{fig:arbreConnaissance}.

\paragraph{Opérations de modifications} Soit deux graphes $G_i(V_i, E_i)$ et
$G_j(V_j, E_j)$, on va définir la fonction \gls{diff} qui renvoie les
informations nécessaires pour passer du graphe $G_i$ au graphe $G_j$. On pose
les ensembles suivants :

\begin{itemize}
  \item $V_{commun} = V_i \cap V_j$
  \item $E_{commun} = E_i \cap E_j$
  \item $V_{supprime} = V_i \setminus V_{commun}$
  \item $E_{supprime} = E_i \setminus E_{commun}$
  \item $V_{ajoute} = V_j \setminus V_{commun}$
  \item $E_{ajoute} = E_j \setminus E_{commun}$
\end{itemize}

Les ensembles $V_{supprime}$, $V_{ajoute}$, $E_{supprime}$ et $E_{ajoute}$
permettent de trouver les opérations pour passer d'un graphe à l'autre. On pose
\gls{diff} la fonction qui pour deux graphes en entré, renvoie ces quatre
ensembles. Précisément $\diff(G_i,G_j) = (V_i \setminus V_j, E_i \setminus E_j,
V_j \setminus V_i, E_j \setminus E_i)$

\paragraph{Arbre des variations de connaissances} L'\gls{avac} \gls{F'} est
identique à l'\gls{avec} \gls{F} sauf que dans chaque nœud, $G_k$ est remplacé
par $\diff(\parent(G_k),G_k)$.

\paragraph{Algorithmes de construction et de traitement} On va maintenant
définir les algorithmes qu'on veut utiliser avec notre structure. Pour cela nous
avons besoin de certaines structures et fonctions qui nous servent pour
interagir avec la simulation. Soit la collection d'agents \gls{A}, ainsi qu'un
ensemble d'instants \gls{T}. On suppose disposer d'une fonction \gls{connait}
qui renvoie la connaissance d'un agent $a\in A$ possède à l'instant $t\in T$ et
d'une fonction \gls{majc} qui met à jour la connaissance d'un agent $a \in A$, à
l'instant $t\in T$.

Les algorithmes à écrire :

\begin{itemize}
  \item Initialisation du graphe des versions de connaissances
  \item Initialisation de l'arbre des versions de connaissances
  \item Initialisation de l'arbre des variations de connaissances
  \item Ajout d'une nouvelle version de connaissance
  \item Suppression d'une version de connaissance \comment{Pas fait}
  \item Dijkstra
  \item Algorithme principal de la simulation
\end{itemize}

\begin{algorithm}%{{{initGVEC
  \caption{Construit le graphe des versions de connaissances à partir de la
  l'ensemble des versions de connaissances}
  \begin{algorithmic}
    \Require $C$, \gls{poids}
    \Function{initGVEC}{$C$}
      \State $E \gets \emptyset$
      \For{$(G_i,G_j) \in C \times C, i < j$}{}
        \State $e = (G_i, G_j, \poids(G_i,G_j))$
        \State $E \gets E \cup \set{e}$
      \EndFor
      \State \Return \gls{G}
    \EndFunction
  \end{algorithmic}
\end{algorithm}%}}}

\begin{algorithm}%{{{initAVEC
  \caption{Construit l'arbre des versions de connaissances à partir du graphes
  versions de connaissances}
  \begin{algorithmic}
    \Require \gls{G}
    \Function{initAVEC}{\gls{G}}
      \State \gls{F} $\gets$ Arbre couvrant de poids minimal de \gls{G}
      \State \Return \gls{F}
    \EndFunction
  \end{algorithmic}
\end{algorithm}%}}}

\begin{algorithm}%{{{initAVAC
  \caption{Construit l'arbre des variations de connaissances à partir de l'arbre
  des versions de connaissances}
  \begin{algorithmic}
    \Require \gls{F}, $\diff$
    \Function{initAVAC}{\gls{F}}
      \Function{ParcoursEnProfondeur}{$c_{courant}, c_{parent}, c'_{parent},
      C'_a, E'_a$}
        \State $c'_{courant} \gets \diff(c_{parent}, c_{courant})$
        \State $C'_a \gets C'_a \cup \set{c'_{courant}}$
        \State $E'_a \gets E'_a \cup \set{(c'_{parent}, c'_{courant})}$
        \For{$c_i \in \enfants(c_{courant})$}{}
          \State $C'_a, E'_a \gets ParcoursEnProfondeur(c_i, c_{courant},
          c'_{courant}, C'_a, E'_a)$
        \EndFor
        \State \Return $C'_a, E'_a$
      \EndFunction
      \State $C', E'_F \gets$ \Call{ParcoursEnProfondeur}{Racine de \gls{F},
      $\emptyset$, $\emptyset$, $\emptyset$, $\emptyset$}
      \State \Return \gls{F'}
    \EndFunction
  \end{algorithmic}
\end{algorithm}%}}}

\begin{algorithm}%{{{nouvelleConnaissance
  \caption{Modifie l'\gls{avac} lorsque la connaissance d'un agent évolue}
  \begin{algorithmic}
    \Require \gls{F'}, $a$ un agent, $G'_d$ La modification de
    connaissance que l'agent découvre, $t$ l'instant avant lequel l'agent a fait
    sa découverte.
    \Comment{Cette procedure produit des effets de bord : elle modifie la
    structure \gls{F'}, et elle appelle la fonction $\majc$.}
    \Procedure{nouvelleConnaissance}{$a, t, G'_d$}
      \State $G'_k \gets \connait(a,t)$
      \State $found \gets False$ \Comment{Vrai si la modification a été trouvé}
      \For{$G'_i \in \enfants(G'_k)$} \Comment{Parmi toutes les modifications}
        \If{$G'_d = G'_i$} \Comment{Si on trouve la modification découverte}
          \State $\majc(a, t+1, G'_i)$ \Comment{On met à jour la connaissance
          dans l'agent}
          \State $found \gets True$
        \EndIf
      \EndFor
      \If{$found = False$} \Comment{Si la modification ne fait pas déjà partie
      de l'arbre}
        \State $C' \gets C' \cup \set{G'_d}$ \Comment{On ajoute la nouvelle
        modification}
        \State $E'_F \gets E'_F \cup \set{(G'_k, G'_i)}$
        \State $\majc(a, t+1, G'_d)$
      \EndIf
    \EndProcedure
  \end{algorithmic}
\end{algorithm}%}}}

\begin{algorithm}%{{{Dijkstra
  \caption{Dijkstra}
  \begin{algorithmic}
    \Require \gls{F'}
    \Function{Dijkstra}{$a$, $t$, $b$ le nœud de départ, $e$ le nœud d'arrivé}
      \State $P \gets \emptyset$
      \State $\forall v \in$ \gls{Vinf}, $d[v] \gets +\infty$
      \State $d[b] \gets 0$
      \State $\forall v \in$ \gls{Vinf}, $predecesseur[v] \gets \emptyset$
      \State $G'_a \gets \connait(a,t)$
      \State $P_G' \gets (\emptyset, G'_1 \dots G'_a)$ chemin des variations de
      connaissances qui part de la racine et arrive à $G'_a$ dans l'arbre
      \gls{F'} (\gls{avac}).
      \While{$\exists v \notin P$}
        \State $v \gets$ Choisir un nœud en dehors de $P$ et avec $d[v]$ minimal
        \State $P \gets P \cup \set{v}$
        \State $N_v \gets \emptyset$
        \For{$G'_k \in P_G'$}
          \State $N_v \gets N_v \cup \set{\text{Les voisins que $G'_k$ ajoute}}$
          \State $N_v \gets N_v \setminus \set{\text{Les voisins que $G'_k$
          enlève}}$
        \EndFor
        \For{$s \in N_v$ et $s \notin P$}
          \If{$d[s] > d[v] + \dist(v,s)$}
            \State $d[s] \gets d[v]+\dist(v,s)$
            \State $predecesseur[s] \gets v$
          \EndIf
        \EndFor
      \EndWhile
      \State $path \gets (e)$
      \While{$predecesseur[path[0]] \neq \emptyset$}
        \State $path \gets predecesseur[path[0]] + path$
      \EndWhile
      \State \Return $path$
    \EndFunction
  \end{algorithmic}
\end{algorithm}%}}}

\begin{algorithm}%{{{Algorithme principal
  \caption{Algorithme principal}
  \begin{algorithmic}
    \State Initialiser les agents $A$
    \State Initialiser les instants $T$
    \State Initialiser les connaissances $C$
    \State \gls{G} $\gets$ \Call{initGVEC}{$C$}
    \State \gls{F} $\gets$ \Call{initAVEC}{\gls{G}}
    \State \gls{F'} $\gets$ \Call{initAVAC}{\gls{F}}
    \For{$t \in T$}
      \For{$a \in A$}
        \If{$a$ ne sait pas quel chemin emprunter}
          \State $path$ = \Call{Dijkstra}{$a, t$, le nœud où se trouve $a$, le
          nœud où $a$ veut aller}
          \State Stocker dans l'agent le chemin ($path$) à emprunter
        \EndIf
        \State Faire avancer l'agent $a$
        \If{$a$ découvre une modification de connaissance $G'_d$}
          \State \Call{nouvelleConnaissance}{$a, t, G'_d$}
        \EndIf
      \EndFor
    \EndFor
  \end{algorithmic}
\end{algorithm}%}}}

\section{Conclusion}

On a cherché à modéliser la connaissance de l'espace de la ville que peuvent
avoir plusieurs agents dans une simulation multi-agents. L'idée est de faire une
modélisation aussi proche de la réalité que possible tout en étant suffisamment
peu gourmand en ressource pour être utilisable dans une simulation temps réel.

On a donc commencé par explicité une solution naïve qui est très proche de la
réalité, mais dont la complexité en temps et en mémoire et si grande qu'il n'est
pas envisageable de l'implémenter. Ainsi on a proposés plusieurs améliorations
envisageables. On discerne deux grandes catégories d'améliorations, celles qui
changent la façon dont on modélise l'espace et celle qui change la façon dont on
organise les connaissances des agents. Ces deux catégories d'améliorations ne
sont pas exclusives, on peut combiner une amélioration qui change la
modélisation de l'espace avec une amélioration qui change l'organisation des
connaissances des agents.

Dans la catégorie des améliorations de la modélisation de l'espace, on a évoqué
une stratégie consistant à utiliser un graphe où les rues sont représentés par
les nœuds et deux nœuds sont connectés si les rues qu'ils représentent
s'intersectent. On a aussi vu qu'on pourrait découper l'espace en quartier et
utiliser des principes de réseaux multi échelles pour diminuer le nombre de
calculs.

Dans la catégorie des améliorations traitant de l'organisation de la
connaissances des agents, nous avons vu deux idées qui se combinent en une
troisième. Ces deux idées sont d'une part la mise en commun de la connaissance,
c'est-à-dire qu'on stocke toutes les versions de connaissances dans une
structure, puis on indique dans chaque agent quel version de connaissance il
possède. Cela évite de stocker deux fois la même version de connaissance lorsque
deux agents ont la même connaissance. D'autre part on propose de ne pas stocker
les connaissances entières dans chaque agent, mais plutôt de ne stocker que la
différence entre une connaissance commune et la connaissance de l'agent.

C'est deux idées se combinent en une structure que l'on a appelé l'\gls{avac}On
s'est particulièrement intéressé à cette organisation de la connaissances des
agents. L'\gls{avac} consiste en un arbre où l'on encode les variations entre
les connaissances des agents plutôt que d'encoder leurs connaissances entières.
On a donné les algorithmes qui, avec les connaissances, permettent de construire
cet arbre, ainsi que deux algorithmes utiles durant la simulation. L'un permet
de modifier la structure lorsqu'un agent découvre nouvel espace. L'autre permet
d'utiliser la structure pour calculer le plus court chemins connu d'un agent
pour aller d'un point à un autre. On donne aussi un exemple sommaire
d'algorithme de simulation utilisant cette structure.

Cependant il reste des zones d'ombres à explorer à propos de l'\gls{avac}. On
doit encore travailler sur la façon dont on calcul la différence entre deux
graphes (i.e., les fonctions \gls{diff} et \gls{poids}). Il existe beaucoup
d'études sur la fonction \gls{poids}, en effet elle est beaucoup utilisé pour
calculer la similarité entre deux graphes. Ce qui est notamment utile pour faire
de la reconnaissance de motifs. Il faudrait faire un état de l'art pour savoir
quelle complexité ces calculs vont avoir dans la situation où l'on se
trouve.\comment{Il me semble qu'on est dans un cas particulier du calcul de la
GED, est-ce que je le dis, même si je ne suis pas sûr ?}

On doit aussi voir comment on veut calculer l'arbre couvrant minimal. Il existe
plusieurs algorithmes, dont certains plus rapide que d'autres. Il y a déjà les
algorithmes gloutons comme l'algorithme de Borůvka qui ont une complexité en
$O(A\ln(S))$ avec $A$ le nombre d'arêtes et $S$ le nombre de sommets. Il existe
aussi des algorithmes qui ont des complexités linéaires avec le nombre d'arêtes,
avec certaines contraintes sur le graphe de départ. Le soucis est que le nombre
d'arêtes dans notre graphe est assez grand puisqu'on a un graphe dense. Il faut
aussi considérer que les poids qui sont dur les arêtes entre les nœuds sont des
poids, ils respectent donc l'inégalité triangulaire. Ces inégalités nous donne
des informations qui peuvent être utile pour calculer l'arbre couvrant de poids
minimal.

On doit s'intéresser à la complexité. Pour l'instant aucune implémentation n'a
été faite et aucune étude précise de la complexité des algorithmes à appliquer
sur la structure. Il faut un cadre théorique pour mesurer les différentes
complexité des solutions. Il faut aussi vérifier ces complexités théoriques par
une implémentation pratique et une batterie de tests. \comment{Étoffer ce
paragraphe}

\printnoidxglossary[sort=def]

\bibliographystyle{apalike} \bibliography{biblio.bib}

\end{document}
